\documentclass[12pt]{article}
%% Generated by Vim reStructured Text 140 - Vim 7.3
\usepackage[a4paper,margin=2.5cm,nohead]{geometry}
\usepackage{enumitem}
\usepackage{graphicx}
\usepackage{longtable}
\usepackage{tabularx}
\usepackage{amsmath}
\usepackage[latin1]{inputenc}

\newenvironment{deflist}[1]{%
\begin{list}{}
{\renewcommand{\makelabel}[1]{\textbf{##1}\hfill}
\settowidth{\labelwidth}{\textbf{#1}}
\leftmargin=\labelwidth
\advance \leftmargin\labelsep}}
{\end{list}}
\newenvironment{optlist}[1]{%
\begin{list}{}
{\renewcommand{\makelabel}[1]{\texttt{##1}\hfill}
\settowidth{\labelwidth}{\texttt{#1}}
\leftmargin=\labelwidth
\advance \leftmargin\labelsep}}
{\end{list}}
\newenvironment{pullquote}{\begin{quotation}\Large}{\end{quotation}}
\setlength{\extrarowheight}{2pt}
\setcounter{tocdepth}{3}
\newcommand{\transition}{\begin{center}\rule{.8\textwidth}{0.2pt}\end{center}}
\newcommand{\subtitle}[1]{{\large\textsc{#1}}\vskip15pt}
\newcommand{\subs}[1]{\raisebox{-0.7ex}{\footnotesize #1}}
\newcommand{\sups}[1]{\raisebox{0.7ex}{\footnotesize #1}}
\newcommand{\attribution}[1]{\raggedleft\textit{#1}}
\newcommand{\rubric}[1]{\vskip15pt{\large #1}}
%% Commands for content of roles directives

%% Commands for content of container directives
\newcommand{\vstvstcompound}[1]{#1}

\usepackage[pdftex]{hyperref}

\hypersetup{pdfcreator={VST, LaTeX, hyperref},
bookmarksopen=true,
bookmarksopenlevel=2,
colorlinks=true,urlcolor=blue,
pdfauthor={Mikolaj Machowski},
pdftitle={Vim reStructured Text},
pdfkeywords={Vim, LaTeX, PDF, HTML, XML},
}

\author{Mikolaj Machowski}
\date{4 Nov 2006}

\begin{document}
\title{Vim reStructured \TeX{}t}
\maketitle
\begin{deflist}{Keywords:}

\item[Author:] Mikolaj Machowski

\item[Title:] Vim reStructured \TeX{}t - HTML and \LaTeX{} output

\item[Keywords:] Vim, \LaTeX{}, PDF, HTML, XML

\item[Version:] 1.4

\item[License:] GPL v. 2

\item[Date:] 4 Nov 2006
\end{deflist}

For a long time \href{http://www.vim.org}{Vim} users were asking for "real" export to HTML. This is,
I believe, first real try to achieve this effect. This is Vim version of
\href{http://docutils.sf.net}{reStructured\TeX{}t}, popular \href{http://www.python.org}{Python} language documentation tool (so I borrowed
parts of its documentation).

\tableofcontents
\hypertarget{lintroduction}{}
\section{Introduction}

\emph{Structured} means Vim script recognizes some patterns and translates them
into form recognizable by WWW browsers. In fact, Vim creates quasi-XML
form which can be exported into HTML and \LaTeX{}.

Major office suites \href{http://www.openoffice.org}{OpenOffice.org} and \href{http://www.microsoft.com}{MS-Office} can import HTML
(\href{http://koffice.kde.org}{KOffice} also can do this but results aren't good) and save files in
their native formats.

\hypertarget{ldownload}{}
\section{Download}
\begin{itemize}
\item
Scripts themselves and docs in txt form: \href{http://skawina.eu.org/mikolaj/vst.zip}{vst.zip}
\end{itemize}

All forms of documentation were created from the same txt source without
\textbf{any} corrections:

\begin{itemize}
\item
HTML documentation: \href{http://skawina.eu.org/mikolaj/vst.html}{vst.html}

\item
PDF documentation: \href{http://skawina.eu.org/mikolaj/vst.pdf}{vst.pdf}

\item
\LaTeX{} source of documentation: \href{http://skawina.eu.org/mikolaj/vst.TeX}{vst.\TeX{}}
\end{itemize}

\hypertarget{larchive}{}

\hypertarget{linstallation}{}
\section{Installation}

Put archive \texttt{vst.zip} into your \texttt{\~{}/.vim} (Unices) or \texttt{vimfiles}
(MS-Windows) directory and unpack it there. It will create files:

\begin{ttfamily}\begin{flushleft}
\mbox{~|-~doc2/vst.txt~~~~~~~~~~~<-~documentation}\\
\mbox{~|-~doc2/vst-s5.txt~~~~~~~~<-~example~of~S5~presentation}\\
\mbox{~|-~doc2/test.png~~~~~~~~~~<-~to~complete~documentation}\\
\mbox{~|-~plugin/vstplugin.vim~~~<-~small~plugin~file~with~commands}\\
\mbox{~|-~autoload/vst/vst.vim~~~<-~main~script}\\
\mbox{~|-~autoload/vst/myhtmlvst.vim~<-~examples~of~macros}\\
\mbox{~$\backslash$-~autoload/vst/s5ui/*~~~~<-~S5~style~files}\\
\end{flushleft}\end{ttfamily}

\hypertarget{lusage}{}
\section{Usage}

Transformation is called with command:

\begin{ttfamily}\begin{flushleft}
\mbox{~:Vst~[\{format\}]}\\
\end{flushleft}\end{ttfamily}

Where \texttt{\{format\}} is format of exported file. Format argument is
optional and without it default value (HTML) will be used. Argument name
is case-insensitive: HTML, html, Html are equivalents. Formatted file
will be opened in new buffer without name.

\begin{deflist}{iii}

\item[\texttt{g:vst\_write\_export}]

Boolean (0). If true, write file immediately with extension specific to
export format (html, \TeX{}, xml). Overwrite existing files without warning.
How to set it read \texttt{:help :let}.
\end{deflist}

Second command:

\begin{ttfamily}\begin{flushleft}
\mbox{~:Vsti~[\{format\}]}\\
\end{flushleft}\end{ttfamily}

Instantly writes file with extension specific to export, overwrite existing
files without warning.

Third command:

\begin{ttfamily}\begin{flushleft}
\mbox{~:Vstm}\\
\end{flushleft}\end{ttfamily}

Is calling menus.

\hypertarget{lproject-file}{}
\subsection{Project file}

User can write variables specific to project into special file \texttt{vstrc.vim},
located into the same directory as processed file. This is regular Vim script
file which will be sourced at the time of export.

\hypertarget{lexport-to-html}{}
\subsection{Export to HTML}

These commands will create XHTML 1.0 file:

\begin{ttfamily}\begin{flushleft}
\mbox{~:Vst~html}\\
\mbox{~:Vst}\\
\end{flushleft}\end{ttfamily}

Vst command (with any export argument) accepts range:

\begin{ttfamily}\begin{flushleft}
\mbox{~:[range]Vst~\{[export]\}}\\
\end{flushleft}\end{ttfamily}

When file use exclusively lists starting from 1/a/A/etc. doctype is Strict, in
other case it is Transitional.

From special characters/entities Vim reStructured \TeX{}t handles at the moment:

\begin{ttfamily}\begin{flushleft}
\mbox{~<,~>,~\&,~(c)}\\
\end{flushleft}\end{ttfamily}

Results in: $<$, $>$, \&, (c).

List can be extended on request.

\hypertarget{lvhp}{}

\begin{deflist}{iii}

\item[\texttt{g:vst\_html\_post}]

String (empty). Filename, sourced after whole processing. In that file
user can put specialized formatting commands, replacing custom or export
dependent templates from replacements etc. For examples check \href{\#lmacros}{macros}
section. How to set it read \texttt{:help :let}.
\end{deflist}
\hypertarget{luser-css}{}
\subsubsection{User CSS}

User can manipulate how CSS will be attached by combinations of these two
variables (how to use Vim variables read \texttt{:help :let}):

\begin{deflist}{iii}

\item[\texttt{g:vst\_css\_default}]

String (empty). When unmodified default CSS will be included in HTML file.
When non-empty default CSS will be written to external file. Existing file
will be overwritten without warning. If equal to NONE (case sensitive) any
reference to default CSS will be skipped.

\item[\texttt{g:vst\_css\_user}]

String (empty). When non-empty link to specified file will be included.
\end{deflist}

Default CSS is in separate file in autoload/vst/default.css so it is easy to
make global-local modifications.

\hypertarget{lexport-to-s5}{}
\subsection{Export to S5}

Command:

\begin{ttfamily}\begin{flushleft}
\mbox{~:Vst~s5}\\
\end{flushleft}\end{ttfamily}

Will create \href{http://meyerweb.com/eric/tools/s5}{S5} -- Simple Standards Based Slide Show System. XHTML file steered
by JavaScript and CSS created by Eric Meyer.

Not every document will be good S5 presentation. All elements but table of
contents are supported but not every one will be good thing to place in one
screen.

\hypertarget{ls5-document}{}
\subsubsection{S5 document}

File should begin with first level header. It will be title of document. Each
slide will begin with second level header. Author and date to place in footer
will be taken from appropriate fields of field list.

\begin{center}
\fbox{\begin{minipage}{0.8\textwidth}
\textbf{\sffamily\large Tip}
\vspace{2mm}

 \begin{itemize}
\item
it is unwise to use more than two levels of headers
\end{itemize}
\end{minipage}}
\end{center}

Vim reStructured \TeX{}t provides predefined classes. Both can be introduced by \href{\#lclass}{class} or
\href{\#lcontainer}{container} directives:

\begin{deflist}{iii}

\item[\texttt{handout} ]

Elements marked with that class will not be seen in presentation mode,
only handout mode.

\item[\texttt{incremental}]

Elements marked with that class will be displayed gradually, when using
normal forward. This one can be also introduced as a~role -- usage of this
element is easier with \href{\#ldefault-role}{default role} directive.
\end{deflist}

Good example of S5 document was prepared by George V.~Reilly and its \TeX{}t
form is included in tarball (doc2/vst-s5.txt). Result can be viewed \href{http://skawina.eu.org/mikolaj/vst-s5.html}{here}.

\hypertarget{lexport-to-LaTeX}{}
\subsection{Export to \LaTeX{}}

These commands will \LaTeX{} version of Vim reStructured \TeX{}t file:

\begin{ttfamily}\begin{flushleft}
\mbox{~:Vst~\LaTeX{}}\\
\mbox{~:Vst~\TeX{}}\\
\end{flushleft}\end{ttfamily}

Document will use \texttt{hyperref} package and it will be fully hyperlinked.
It is better to compile it with \texttt{pdf\LaTeX{}}, directly to PDF than use
pure \texttt{\LaTeX{}}.

\hypertarget{lvtp}{}

\begin{deflist}{iii}

\item[\texttt{g:vst\_\TeX{}\_preamble}]

String (empty). User can define his own command (to use in
\href{\#lraw-LaTeX}{raw \LaTeX{}}) or change some defaults. Should be absolute or relative
path to processed file. Contents of file will be included with
\texttt{$\backslash$input\{\}}.

\item[\texttt{g:vst\_\TeX{}\_post}]

String (empty). Filename, sourced after whole processing. In that file
user can put specialized formatting commands, replacing custom or export
dependent templates from replacements etc.
\end{deflist}

Current differences between HTML and \LaTeX{} export:

\begin{itemize}
\item
2html is treated as ordinary preformatted \TeX{}t

\item
custom roles are ignored

\item
right/left floating frames are ignored
\end{itemize}
\hypertarget{lexport-to-pdf}{}
\subsection{Export to PDF}

This command will immediately compile PDF document after creation of
\LaTeX{} source:

\begin{ttfamily}\begin{flushleft}
\mbox{~:Vst~pdf}\\
\end{flushleft}\end{ttfamily}

Additional tuning of this command with variables (how to use variables read
\texttt{:help :let}):

\begin{deflist}{iii}

\item[\texttt{g:vst\_pdf\_view}]

Boolean (0). When equal 1 immediately show result of compilation.

\item[\texttt{g:vst\_pdf\_viewer}]

String (default for unices is \texttt{xpdf}, for MS-Windows empty
-- properly setup system will take care about choosing application).

\item[\texttt{g:vst\_pdf\_clean}]

Boolean (0). When equal 1 remove auxiliary files of PDF compilation
(including \TeX{} source -- use with caution!)

\item[\texttt{g:vst\_pdf\_command}]

String (\texttt{pdf\LaTeX{} -interaction=nonstopmode}). Value of this string will
be used to produce PDF document. User may need to modify this variable,
especially if document needs some preprocessing.
\end{deflist}
\hypertarget{lexport-to-xml}{}
\subsection{Export to XML}

This command will produce XML-like code:

\begin{ttfamily}\begin{flushleft}
\mbox{~:Vst~xml}\\
\end{flushleft}\end{ttfamily}

For debugging purposes.

\hypertarget{lexport-to-rest}{}
\subsection{Export to reST}

This command will remove incompatibilities with reST:

\begin{ttfamily}\begin{flushleft}
\mbox{~:Vst~rest}\\
\end{flushleft}\end{ttfamily}

\begin{itemize}
\item
identify option of \href{\#limages}{images} and \href{\#lfigures}{figures} will be deleted

\item
figalign option of \href{\#lfigures}{figures} will be deleted

\item
in Vim commands of \href{\#loption-list}{option list} leading colon will be replaced with
\texttt{--VIM, :}

\item
replace non-breaking space with plain space

\item
replace 2html directive declaration with '::'
\end{itemize}
\begin{center}
\fbox{\begin{minipage}{0.8\textwidth}
\textbf{\sffamily\large Warning}
\vspace{2mm}

This export can result in losing of some formatting.
\end{minipage}}
\end{center}
\hypertarget{lauxiliary-commands}{}
\subsection{Auxiliary commands}

Vim reStructured \TeX{}t script provides set of auxiliary commands to make easier editing
of bigger files and especially navigating through them.

\hypertarget{lheaders}{}

\hypertarget{l9696head9696}{}
\subsubsection{\texttt{head}}

It is possible to lost orientation which type of underscore was used to
mark certain level of sections:

\begin{ttfamily}\begin{flushleft}
\mbox{~:Vst~head}\\
\end{flushleft}\end{ttfamily}

Will show small table with level name and symbols used to mark them:

\begin{ttfamily}\begin{flushleft}
\mbox{~Level~~~~~~~~~~~~~~~Symbol}\\
\mbox{~h1~~~~~~~~~~~~~~~~~~=========d}\\
\mbox{~h2~~~~~~~~~~~~~~~~~~---------~}\\
\mbox{~h3~~~~~~~~~~~~~~~~~~\~{}\~{}\~{}\~{}\~{}\~{}\~{}\~{}\~{}~}\\
\mbox{~h4~~~~~~~~~~~~~~~~~~+++++++++}\\
\end{flushleft}\end{ttfamily}

This is result of \texttt{Vst head} for this manual. Letter \texttt{d} at the end of
ornament shows this is double ornament.

\hypertarget{lcontents}{}

\hypertarget{l9696toc9696}{}
\subsubsection{\texttt{toc}}

In longer \TeX{}t documents it is very easy to be lost:

\begin{ttfamily}\begin{flushleft}
\mbox{~:Vst~toc}\\
\end{flushleft}\end{ttfamily}

This command will present table of contents for current document. Here
is fragment of table of contents of this manual:

\begin{ttfamily}\begin{flushleft}
\mbox{~h3~\~{}\~{}\~{}\~{}\~{}\~{}\~{}\~{}\~{}~~~5.3~~Inline~markup~rules~~~~~~~~~~~~~~~~553}\\
\mbox{~h3~\~{}\~{}\~{}\~{}\~{}\~{}\~{}\~{}\~{}~~~5.4~~Special~characters~~~~~~~~~~~~~~~~~587}\\
\mbox{~h3~\~{}\~{}\~{}\~{}\~{}\~{}\~{}\~{}\~{}~~~[[[~5.5~~Roles~]]]~~~~~~~~~~~~~~~~~~~~~~625}\\
\mbox{~h4~`````````~~~5.5.1~~Predefined~roles~~~~~~~~~~~~~~~~~652}\\
\mbox{~h5~'''''''''~~~5.5.1.1~~Title~reference~~~~~~~~~~~~~~~~658}\\
\mbox{~h5~'''''''''~~~5.5.1.2~~Subscript~~~~~~~~~~~~~~~~~~~~~~670}\\
\end{flushleft}\end{ttfamily}

It gives plenty of information. First column is name of section level; second
column shows decoration for that level; third column are section titles with
numbers of sections as they will be shown in HTML and \LaTeX{} export. Title of
section were cursor was at the moment of invoking command is taken into 
\texttt{[[[ ]]]}; fourth column are line numbers of section titles -- user can
immediately issue Vim command, eg. \texttt{:625} to go to Roles section.

\hypertarget{l9696link9696-and-9696slink9696}{}
\subsubsection{\texttt{link} and \texttt{slink}}

Commands will show all external links and internal explicit targets in
chronological order:

\begin{ttfamily}\begin{flushleft}
\mbox{~:Vst~link}\\
\end{flushleft}\end{ttfamily}

In alphabetical order:

\begin{ttfamily}\begin{flushleft}
\mbox{~:Vst~slink}\\
\end{flushleft}\end{ttfamily}

Fragment of link table for this document (in chronological order):

\begin{ttfamily}\begin{flushleft}
\mbox{~\TeX{}t~~~~~~~~~~~~~~~~Link}\\
\mbox{~Vim~~~~~~~~~~~~~~~~~http://www.vim.org}\\
\mbox{~reStructured\TeX{}t~~~~http://docutils.sf.net}\\
\mbox{~starting~point~~~~~~http://www.vim.org}\\
\mbox{~comment}\\
\mbox{~Opera~~~~~~~~~~~~~~~http://www.opera.com}\\
\mbox{~vst.pdf~~~~~~~~~~~~~http://skawina.eu.org/mikolaj/vst.pdf}\\
\end{flushleft}\end{ttfamily}

Item without Link part (second column) is internal explicit target.

\hypertarget{l9696rep9696-and-9696srep9696}{}
\subsubsection{\texttt{rep} and \texttt{srep}}

Commands will show replace declarations. In chronological order:

\begin{ttfamily}\begin{flushleft}
\mbox{~:Vst~rep}\\
\end{flushleft}\end{ttfamily}

And in alphabetical:

\begin{ttfamily}\begin{flushleft}
\mbox{~:Vst~srep}\\
\end{flushleft}\end{ttfamily}

Replace table for this document (in chronological order):

\begin{ttfamily}\begin{flushleft}
\mbox{~Symbol~~~~~~~~~~~~~~Replacement}\\
\mbox{~H2O~~~~~~~~~~~~~~~~~H$\backslash$~:sub:`2`$\backslash$~O}\\
\mbox{~from~~~~~~~~~~~~~~~~image:image.jpg~width:20~height:20~target:URL}\\
\mbox{~VST~~~~~~~~~~~~~~~~~Vim~reStructured~\TeX{}t}\\
\end{flushleft}\end{ttfamily}

\hypertarget{lfolding}{}
\subsubsection{Folding}

Vim reStructured \TeX{}t offers whole family of commands for folding of file.

This command will create simple folds for document, each section has its own
fold on the same level:

\begin{ttfamily}\begin{flushleft}
\mbox{~:Vst~fold}\\
\end{flushleft}\end{ttfamily}

Folds are created based on regular expressions and when you add new \TeX{}t to
section fold will be expanded to include them. New sections won't be
recognized automatically and you should recreate folds to include new
sections.

In front of header \TeX{}t you will see numbers of section as they will be
visible in HTML and \LaTeX{} formats.

At the end of line enclosed in parenthesis are charaters used as ornament of
section title.

Command:

\begin{ttfamily}\begin{flushleft}
\mbox{~:Vst~foldr}\\
\end{flushleft}\end{ttfamily}

Will create folds recursively, each level of headers will have its own level
of folding.

Commands:

\begin{ttfamily}\begin{flushleft}
\mbox{~:Vst~fold1}\\
\mbox{~:Vst~fold2}\\
\mbox{~:Vst~fold3}\\
\mbox{~:Vst~fold4}\\
\mbox{~:Vst~fold5}\\
\end{flushleft}\end{ttfamily}

Will create one level of folds down to this level of headers. This is can be
useful for visualization of table of contents and faster navigation by big
structures of \TeX{}t.

\hypertarget{lfolding-by-directive}{}
\subsubsection{Folding by directive}

For easier finding of some directives user can use special case of folding:

\begin{ttfamily}\begin{flushleft}
\mbox{~:Vst~f\{directive\}}\\
\end{flushleft}\end{ttfamily}

Where \texttt{\{directive\}} is name of directive, eg.: container, blockquote,
image, figure, tip, note, etc. Directive name is case insensitive.

Special case is:

\begin{ttfamily}\begin{flushleft}
\mbox{~:Vst~fblank}\\
\end{flushleft}\end{ttfamily}

Document will be folded by paragraphs -- fragments of \TeX{}t separated by
blank lines.

\hypertarget{l9696help9696}{}
\subsubsection{\texttt{help}}

This command will show short summary of Vim reStructured \TeX{}t commands:

\begin{ttfamily}\begin{flushleft}
\mbox{~:Vst~help}\\
\end{flushleft}\end{ttfamily}

\hypertarget{l9696preproc9696}{}
\subsubsection{\texttt{preproc}}

This command will preprocess file and include in file all including commands:

\begin{ttfamily}\begin{flushleft}
\mbox{~:Vst~preproc}\\
\end{flushleft}\end{ttfamily}

This command is recursive. For details see \href{\#lincluding-commands}{including commands}

\hypertarget{lornament-mapping}{}
\subsubsection{Ornament mapping}

Manual insertion of \href{\#lsections}{sections} or \href{\#ltransitions}{transitions} ornaments is boring. To speed
things up in Vim way auxiliary mapping was created: $<$C-B$>$o.

When placed below line of \TeX{}t and above empty one it will fill line to length
of line above. It may use character before cursor, when not available it will
use last single ornamented section title as hint.

When placed above line of \TeX{}t and below empty line it will embrace line below
in section ornaments. It may use character before cursor, when not available it
will use last double ornamented section title as hint.

When placed between empty lines it will fill it as transition element. It may
use character before cursor, when not available it will use last transition as
hint.

\hypertarget{lvim-settings}{}
\subsection{Vim settings}

Set of Vim settings which can be useful for editing of Vim reStructured \TeX{}t files:

\begin{ttfamily}\begin{flushleft}
\mbox{~set~nocompatible}\\
\mbox{~set~autoindent}\\
\mbox{~set~formatoptions+=tqn}\\
\mbox{~set~formatlistpat=\^{}$\backslash$$\backslash$s*$\backslash$$\backslash$($\backslash$$\backslash$d$\backslash$$\backslash$+$\backslash$$\backslash$$\backslash$|[a-z]$\backslash$$\backslash$)[$\backslash$$\backslash$].)]$\backslash$$\backslash$s*}\\
\mbox{~set~\TeX{}twidth=78~~~~"~purely~arbitrary~value,~just~remember~to~set~it}\\
\mbox{~set~expandtab}\\
\mbox{~set~softtabstop=4~~~"~less~than~4~may~result~in~breaking~of~lists}\\
\mbox{~set~shiftwidth=4}\\
\mbox{~set~formatoptions+=tqn}\\
\end{flushleft}\end{ttfamily}

\hypertarget{lsyntax-highlighting}{}
\subsection{Syntax highlighting}

Structure of document will be more visible with syntax highlighting. It is
possible to use Vim default syntax highlighting file by Nicolai Weibull just
by issuing Vim command:

\begin{ttfamily}\begin{flushleft}
\mbox{~:set~ft=rst}\\
\end{flushleft}\end{ttfamily}

On \href{http://www.vim.org/scripts/script.php?script_id=973}{vim-online} site is another syntax file by Estienne Swart.

You can plase filetype declaration in Vim modeline, it will be removed from
output file to not confuse Vim filetype detection (modelines have the highest
priority). But it will work only if filetype setting is in separate Vim reStructured \TeX{}t
comment, line matching:

\begin{ttfamily}\begin{flushleft}
\mbox{~\^{}$\backslash$s*$\backslash$.$\backslash$.~vim:}\\
\end{flushleft}\end{ttfamily}

This line has to be also inside of 'modelines' Vim option range.

\hypertarget{lmenus}{}
\subsection{Menus}

Command:

\begin{ttfamily}\begin{flushleft}
\mbox{~:Vstm}\\
\end{flushleft}\end{ttfamily}

in GUI version (gvim) will create menu with most common commands:

\begin{deflist}{Keywords:}

\item[Export to HTML:] \href{\#lexport-to-html}{export to HTML}

\item[Export to \LaTeX{}:] \href{\#lexport-to-LaTeX}{export to \LaTeX{}}

\item[Export to PDF:] \href{\#lexport-to-pdf}{export to PDF} 

\item[Export to reST:] \href{\#lexport-to-rest}{export to reST}

\item[Fold:] \href{\#lfolding}{folding} of document

\item[Headers:] show symbols of \href{\#lheaders}{headers} for various levels

\item[TOC:] show \href{\#lcontents}{contents} of \TeX{}t document

\item[Help:] show short help at the bottom of screen
\end{deflist}

To show menu always put this command in .vimrc:

\begin{deflist}{iii}

\item[\texttt{g:vst\_showmenu}]

Boolean (0). When 1 and set in .vimrc it will always show menu.
\end{deflist}
\hypertarget{lstructure}{}
\section{Structure}
\hypertarget{lparagraphs}{}
\subsection{Paragraphs}

Base unit of \TeX{}t is \textbf{paragraph}, \TeX{}t separated by at least one blank
line. All lines in paragraphs must have the same indentation. Paragraph
indented will be displayed as quotation (blockquote). It is possible to
embed any type and any number of elements inside blockquote -- respecting
their own rules of embedding. Example:

\begin{ttfamily}\begin{flushleft}
\mbox{~This~is~example~of~paragraph.~This}\\
\mbox{~is~continuation~of~paragraph.}\\
\mbox{}\\
\mbox{~~~~~This~is~indented~paragraph.~Looking}\\
\mbox{~~~~~like~quoted~\TeX{}t.}\\
\mbox{}\\
\mbox{~~~~~~~~~This~is~quoted~quoted~\TeX{}t.~Indented~two~times.}\\
\mbox{}\\
\mbox{~This~is~another~one.}\\
\end{flushleft}\end{ttfamily}

Results in:

This is example of paragraph. This
is continuation of paragraph.

 \begin{quotation}
 This is indented paragraph. Looking
 like quoted \TeX{}t.

 \begin{quotation}
 This is quoted quoted \TeX{}t. Indented two times.
 \end{quotation}
 \end{quotation}

This is another one.

Embedding of elements is supported for paragraphs, blockquotes, ordered
lists, unordered lists and tip, note, warning directives. In all of them
can be embedded the rest of one-level elements.

\hypertarget{lTeXt-styles}{}
\subsection{\TeX{}t styles}

Inside of paragraph (and other \TeX{}t elements you can use another markup
-- \emph{italics} with \texttt{*italics*}, \textbf{bold} with \texttt{**bold**},
\texttt{``double back-quotes``} for \texttt{typewriter \TeX{}t}.

This \TeX{}t is \emph{italicised}.

This \TeX{}t is \textbf{strongly emphasised}.

This \TeX{}t is \texttt{a code}.

If you find that you want to use one of the "special" characters in
\TeX{}t, it should be OK -- Vim reStructured \TeX{}t can deal with most typical situations.
For example, this * asterisk is handled just fine. If you actually
want \TeX{}t *surrounded by asterisks* to \textbf{not} be italicised, then
you need to indicate that the asterisk is not special. You do this by
placing a backslash just before it, like so "\texttt{$\backslash$$\backslash$*}". Remember: special
treatment of these few characters is entering inline literals -- even
there you have to escape it with double backslash:

\begin{ttfamily}\begin{flushleft}
\mbox{~``$\backslash$$\backslash$*``}\\
\end{flushleft}\end{ttfamily}

For another method of font manipulation check \href{\#lroles}{Roles}.

\hypertarget{linline-markup-rules}{}
\subsection{Inline markup rules}

These rules apply for \href{\#lTeXt-styles}{\TeX{}t styles} and all other types of inline markup,
especially \href{\#llinks}{links}.

The inline markup start-string and end-string recognition rules are as
follows. If any of the conditions are not met, the start-string or
end-string will not be recognized or processed.

\begin{enumerate}[label=\arabic*.]
\item
Inline markup start-strings must start a \TeX{}t block or be
immediately preceded by whitespace or one of:

\begin{ttfamily}\begin{flushleft}
\mbox{~'~"~(~[~\{~<~-~/~:}\\
\end{flushleft}\end{ttfamily}

\item
Inline markup start-strings must be immediately followed by
non-whitespace.

\item
Inline markup end-strings must be immediately preceded by
non-whitespace.

\item
Inline markup end-strings must end a \TeX{}t block or be immediately
followed by whitespace or one of:

\begin{ttfamily}\begin{flushleft}
\mbox{~'~"~)~]~\}~>~-~/~:~.~,~;~!~?~$\backslash$}\\
\end{flushleft}\end{ttfamily}

\item
If an inline markup start-string is immediately preceded by a
single or double quote, "(", "[", "\{", or "$<$", it must not be
immediately followed by the corresponding single or double quote,
")", "]", "\}", or "$>$".

\item
An inline markup end-string must be separated by at least one
character from the start-string.
\end{enumerate}
\hypertarget{lspecial-characters}{}
\subsection{Special characters}

Sometimes it is necessary to escape special treatment of some
characters (or give that meaning). Then you have to put backslash \texttt{$\backslash$}
before them.

Remove space:

\begin{ttfamily}\begin{flushleft}
\mbox{~this$\backslash$~that}\\
\end{flushleft}\end{ttfamily}

Result: thisthat

Do not italicise word:

\begin{ttfamily}\begin{flushleft}
\mbox{~not~$\backslash$*this*~word}\\
\end{flushleft}\end{ttfamily}

Result: not *this* word

Insert non-breaking space:

\begin{ttfamily}\begin{flushleft}
\mbox{~non$\backslash$-breaking$\backslash$-space}\\
\end{flushleft}\end{ttfamily}

Result: non~breaking~space (\texttt{\&nbsp;} in HTML)

\begin{center}
\fbox{\begin{minipage}{0.8\textwidth}
\textbf{\sffamily\large Note}
\vspace{2mm}

This construction should be avoided and use non-breaking space
instead. It will be replaced with tilde in \LaTeX{} export and leaved
as-is in HTML (it is correctly interpreted by browsers.

 \begin{quotation}
 To input non-breaking space in Vim use:

 \begin{itemize}
\item
$<$C-V$>$160 in Latin1 and Latin2 encodings (\texttt{:help i\_CTRL-V\_digit})

\item
$<$C-K$>$NS (\texttt{:help i\_digraph})
\end{itemize}
 \end{quotation}
\end{minipage}}
\end{center}

Backslash should be escaped by putting before it another backslash only in
case when backslash could be used in special character. Slight inconsistency
but generally makes \TeX{}t more readable.

\hypertarget{lroles}{}
\subsection{Roles}

Vim reStructured \TeX{}t supports additional methods of \TeX{}t manipulation. They are called
\emph{roles}. Usual form is:

\begin{ttfamily}\begin{flushleft}
\mbox{~:name:`\TeX{}t`}\\
\end{flushleft}\end{ttfamily}

Roles are requiring white spaces or non-word characters around them. You
can escape them so they will be:

\begin{ttfamily}\begin{flushleft}
\mbox{~H$\backslash$~:sub:`2`$\backslash$~O}\\
\end{flushleft}\end{ttfamily}

Result: H\subs{2}O

It looks awkwardly, especially if repeated many times in \TeX{}t. You can
help it with \href{\#lreplacement}{replacement}:

\begin{ttfamily}\begin{flushleft}
\mbox{~|H2O|}\\
\mbox{}\\
\mbox{~..~|H2O|~replace::~H$\backslash$~:sub:`2`$\backslash$~O}\\
\end{flushleft}\end{ttfamily}

Result is: H\subs{2}O

\hypertarget{lpredefined-roles}{}
\subsubsection{Predefined roles}

All predefined roles can be recognized as arguments for \href{\#ldefault-role}{default role}
directive.

\hypertarget{ltitle-reference}{}
\paragraph{Title reference}

This role will should be used to display \emph{Titles of books} and similar
citation sources. There are three ways to declare it in \TeX{}t:

\begin{ttfamily}\begin{flushleft}
\mbox{~:title-reference:`Title`}\\
\mbox{~:title:`Title`}\\
\mbox{~:t:`Title`}\\
\end{flushleft}\end{ttfamily}

Title reference is default role for interpreted \TeX{}t.

\hypertarget{lsubscript}{}
\paragraph{Subscript}

This role will show \subs{small \TeX{}t below} normal line of \TeX{}t:

\begin{ttfamily}\begin{flushleft}
\mbox{~:sub:`small~\TeX{}t~below`}\\
\end{flushleft}\end{ttfamily}

\hypertarget{lsuperscript}{}
\paragraph{Superscript}

This role will show \sups{small \TeX{}t over} normal line of \TeX{}t:

\begin{ttfamily}\begin{flushleft}
\mbox{~:sup:`small~\TeX{}t~over`}\\
\end{flushleft}\end{ttfamily}

\hypertarget{lbig}{}
\paragraph{Big}

This role will show some {\large bigger \TeX{}t}:

\begin{ttfamily}\begin{flushleft}
\mbox{~:big:`bigger~\TeX{}t`}\\
\end{flushleft}\end{ttfamily}

\hypertarget{lsmall}{}
\paragraph{Small}

This role will show some {\small smaller \TeX{}t}:

\begin{ttfamily}\begin{flushleft}
\mbox{~:small:`smaller~\TeX{}t`}\\
\end{flushleft}\end{ttfamily}

\hypertarget{lstrong}{}
\paragraph{Strong}

This role will show \textbf{bold \TeX{}t}:

\begin{ttfamily}\begin{flushleft}
\mbox{~:strong:`bold~\TeX{}t`}\\
\end{flushleft}\end{ttfamily}

\hypertarget{lemphasis}{}
\paragraph{Emphasis}

This role will show \emph{emphasised \TeX{}t}:

\begin{ttfamily}\begin{flushleft}
\mbox{~:emphasis:`emphasised~\TeX{}t`}\\
\end{flushleft}\end{ttfamily}

\hypertarget{lliteral}{}
\paragraph{Literal}

This role will show \texttt{monospaced \TeX{}t}:

\begin{ttfamily}\begin{flushleft}
\mbox{~:literal:`monospaced~\TeX{}t`}\\
\end{flushleft}\end{ttfamily}

\hypertarget{lcustom-roles}{}
\subsubsection{Custom roles}

You can use custom roles which will be marked in Vim reStructured \TeX{}t source as:

\begin{ttfamily}\begin{flushleft}
\mbox{~:custom:`special~\TeX{}t`}\\
\end{flushleft}\end{ttfamily}

And in HTML output:

\begin{ttfamily}\begin{flushleft}
\mbox{~<span~class="custom">special~\TeX{}t</span>}\\
\end{flushleft}\end{ttfamily}

In \LaTeX{} output:

\begin{ttfamily}\begin{flushleft}
\mbox{~$\backslash$vstcustom\{special~\TeX{}t\}}\\
\end{flushleft}\end{ttfamily}

Note \texttt{vst} prefix to avoid possible conflicts with built-in (La)\TeX{}
commands. In preamble will be inserted simple template to allow compilation of
document without stopping on unknown commands:

\begin{ttfamily}\begin{flushleft}
\mbox{~$\backslash$newcommand\{$\backslash$vstcustom\}[1]\{$\backslash$\TeX{}tnormal\{\#1\}\}}\\
\end{flushleft}\end{ttfamily}

It will be inserted before declaration of extension of preamble by \href{\#lvtp}{external
file}. If user wants to declare there these new commands he should
use \texttt{renewcommand} \LaTeX{} command.

Roles have to be declared through \hypertarget{lrole-directive}{role directive}. When not declared will be
silently ignored and any sign of them will be removed from output. Declaration
of role:

\begin{ttfamily}\begin{flushleft}
\mbox{~..~role::~custom}\\
\end{flushleft}\end{ttfamily}

May be used for example to underline fragment of \TeX{}t (style of decoration not
supported by \href{http://docutils.sf.net}{reST} or Vim reStructured \TeX{}t):

\begin{ttfamily}\begin{flushleft}
\mbox{~..~role::~ul}\\
\mbox{}\\
\mbox{~This~fragment~:ul:`will~be~underlined`}\\
\end{flushleft}\end{ttfamily}

And in \href{\#luser-css}{user CSS} file:

\begin{ttfamily}\begin{flushleft}
\mbox{~span.ul~\{~\TeX{}t-decoration:~underline~\}}\\
\end{flushleft}\end{ttfamily}

Role directive may have an option \texttt{class}:

\begin{ttfamily}\begin{flushleft}
\mbox{~..~role::~name}\\
\mbox{~~~~:class:~special}\\
\end{flushleft}\end{ttfamily}

It will turn:

\begin{ttfamily}\begin{flushleft}
\mbox{~This~is~:name:`wonderful`~feature.}\\
\end{flushleft}\end{ttfamily}

Into:

\begin{ttfamily}\begin{flushleft}
\mbox{~This~is~<span~class="special">wonderful</span>~feature.}\\
\end{flushleft}\end{ttfamily}

Useful when CSS name is long or not meaningful.

\hypertarget{lreversed-notation}{}
\subsubsection{Reversed notation}

Vim reStructured \TeX{}t supports also reversed notation of roles:

\begin{ttfamily}\begin{flushleft}
\mbox{~`\TeX{}t`:name:}\\
\end{flushleft}\end{ttfamily}

But be careful. Use of it in some cases (replacement-link combo) can give
weird results.

\hypertarget{llists}{}
\subsection{Lists}

Lists of items come in three main flavours: \textbf{enumerated}, \textbf{bulleted} and
\textbf{definitions}. List element can contain many body elements.

Lists must always start a new paragraph -- that is, they must appear
after a blank line.

\hypertarget{lenumerated-lists}{}
\subsubsection{Enumerated lists}

Start a line off with a number or letter followed by a period ".",
right bracket ")" or colon ":", also enclosed in parenthesis (a). Following
forms are recognised:

\begin{ttfamily}\begin{flushleft}
\mbox{~1.~numbers}\\
\mbox{}\\
\mbox{~A.~upper-case~letters}\\
\mbox{~~~~and~it~goes~over~many~lines~and~may~contain~body~elements~like}\\
\mbox{}\\
\mbox{~~~~1.~sub-}\\
\mbox{~~~~2.~-lists}\\
\mbox{}\\
\mbox{~~~~::}\\
\mbox{}\\
\mbox{~~~~~~~~Or~preformatted~\TeX{}t}\\
\mbox{}\\
\mbox{~a.~lower-case~letters}\\
\mbox{}\\
\mbox{~1)~numbers~again}\\
\mbox{}\\
\mbox{~i.~lower~roman~numerals}\\
\mbox{}\\
\mbox{~I.~and~upper~roman~literals}\\
\mbox{}\\
\mbox{~(a)~enumerator~enclosed~in~parenthesis}\\
\end{flushleft}\end{ttfamily}

Results in:

\begin{enumerate}[label=\arabic*.]
\item
numbers
\end{enumerate}
\begin{enumerate}[label=\alph*.]
\item
A. upper-case letters
and it goes over many lines and may contain body elements like

 \begin{enumerate}[label=\arabic*.]
\item
sub-

\item
-lists
 \end{enumerate}

\begin{ttfamily}\begin{flushleft}
\mbox{~Or~preformatted~\TeX{}t}\\
\end{flushleft}\end{ttfamily}

 Or many other elements.

\item
lower-case letters
\end{enumerate}
\begin{enumerate}[label=\arabic*.]
\item
numbers again
\end{enumerate}
\begin{enumerate}[label=\Roman*.]
\item 
\begin{enumerate}[label=\Roman*.]
\item 
\begin{enumerate}[label=\Roman*.]
\item 
\begin{enumerate}[label=\Roman*.]
\item 
\begin{enumerate}[label=\Roman*.]
\item 
\begin{enumerate}[label=\Roman*.]
\item 
\begin{enumerate}[label=\Roman*.]
\item 
\begin{enumerate}[label=\Roman*.]
\item 
\begin{enumerate}[label=\Roman*.]
\item 
\begin{enumerate}[label=\Roman*.]
\item 
\begin{enumerate}[label=\Roman*.]
\item 
\begin{enumerate}[label=\Roman*.]
\item 
\begin{enumerate}[label=\Roman*.]
\item 
\begin{enumerate}[label=\Roman*.]
\item 
\begin{enumerate}[label=\Roman*.]
\item 
\begin{enumerate}[label=\Roman*.]
\item 
\begin{enumerate}[label=\Roman*.]
\item 
\begin{enumerate}[label=\Roman*.]
\item 
\begin{enumerate}[label=\Roman*.]
\item 
\begin{enumerate}[label=\Roman*.]
\item 
\begin{enumerate}[label=\Roman*.]
\item 
\begin{enumerate}[label=\Roman*.]
\item 
\begin{enumerate}[label=\Roman*.]
\item 
\begin{enumerate}[label=\Roman*.]
\item 
\begin{enumerate}[label=\Roman*.]
\item 
\begin{enumerate}[label=\Roman*.]
\item 
\begin{enumerate}[label=\Roman*.]
\item 
\begin{enumerate}[label=\Roman*.]
\item 
\begin{enumerate}[label=\Roman*.]
\item 
\begin{enumerate}[label=\Roman*.]
\item 
\begin{enumerate}[label=\Roman*.]
\item 
\begin{enumerate}[label=\Roman*.]
\item 
\begin{enumerate}[label=\Roman*.]
\item 
\begin{enumerate}[label=\Roman*.]
\item 
\begin{enumerate}[label=\Roman*.]
\item 
\begin{enumerate}[label=\Roman*.]
\item 
\begin{enumerate}[label=\Roman*.]
\item 
\begin{enumerate}[label=\Roman*.]
\item 
\begin{enumerate}[label=\Roman*.]
\item 
\begin{enumerate}[label=\Roman*.]
\item 
\begin{enumerate}[label=\Roman*.]
\item 
\begin{enumerate}[label=\Roman*.]
\item 
\begin{enumerate}[label=\Roman*.]
\item 
\begin{enumerate}[label=\Roman*.]
\item 
\begin{enumerate}[label=\Roman*.]
\item 
\begin{enumerate}[label=\Roman*.]
\item 
\begin{enumerate}[label=\Roman*.]
\item 
\begin{enumerate}[label=\Roman*.]
\item 
\begin{enumerate}[label=\Roman*.]
\item 
\begin{enumerate}[label=\Roman*.]
\item 
\begin{enumerate}[label=\Roman*.]
\item 
\begin{enumerate}[label=\Roman*.]
\item 
\begin{enumerate}[label=\Roman*.]
\item 
\begin{enumerate}[label=\Roman*.]
\item 
\begin{enumerate}[label=\Roman*.]
\item 
\begin{enumerate}[label=\Roman*.]
\item 
\begin{enumerate}[label=\Roman*.]
\item 
\begin{enumerate}[label=\Roman*.]
\item 
\begin{enumerate}[label=\Roman*.]
\item 
\begin{enumerate}[label=\Roman*.]
\item 
\begin{enumerate}[label=\Roman*.]
\item 
\begin{enumerate}[label=\Roman*.]
\item 
\begin{enumerate}[label=\Roman*.]
\item 
\begin{enumerate}[label=\Roman*.]
\item 
\begin{enumerate}[label=\Roman*.]
\item 
\begin{enumerate}[label=\Roman*.]
\item 
\begin{enumerate}[label=\Roman*.]
\item 
\begin{enumerate}[label=\Roman*.]
\item 
\begin{enumerate}[label=\Roman*.]
\item 
\begin{enumerate}[label=\Roman*.]
\item 
\begin{enumerate}[label=\Roman*.]
\item 
\begin{enumerate}[label=\Roman*.]
\item 
\begin{enumerate}[label=\Roman*.]
\item 
\begin{enumerate}[label=\Roman*.]
\item 
\begin{enumerate}[label=\Roman*.]
\item 
\begin{enumerate}[label=\Roman*.]
\item 
\begin{enumerate}[label=\Roman*.]
\item 
\begin{enumerate}[label=\Roman*.]
\item 
\begin{enumerate}[label=\Roman*.]
\item 
\begin{enumerate}[label=\Roman*.]
\item 
\begin{enumerate}[label=\Roman*.]
\item 
\begin{enumerate}[label=\Roman*.]
\item 
\begin{enumerate}[label=\Roman*.]
\item 
\begin{enumerate}[label=\Roman*.]
\item 
\begin{enumerate}[label=\Roman*.]
\item 
\begin{enumerate}[label=\Roman*.]
\item 
\begin{enumerate}[label=\Roman*.]
\item 
\begin{enumerate}[label=\Roman*.]
\item 
\begin{enumerate}[label=\Roman*.]
\item 
\begin{enumerate}[label=\Roman*.]
\item 
\begin{enumerate}[label=\Roman*.]
\item 
\begin{enumerate}[label=\Roman*.]
\item 
\begin{enumerate}[label=\Roman*.]
\item 
\begin{enumerate}[label=\Roman*.]
\item 
\begin{enumerate}[label=\Roman*.]
\item 
\begin{enumerate}[label=\Roman*.]
\item 
\begin{enumerate}[label=\Roman*.]
\item 
\begin{enumerate}[label=\Roman*.]
\item 
-1
\item -1
-1
\end{enumerate}
\end{enumerate}
\end{enumerate}
\end{enumerate}
\end{enumerate}
\end{enumerate}
\end{enumerate}
\end{enumerate}
\end{enumerate}
\end{enumerate}
\end{enumerate}
\end{enumerate}
\end{enumerate}
\end{enumerate}
\end{enumerate}
\end{enumerate}
\end{enumerate}
\end{enumerate}
\end{enumerate}
\end{enumerate}
\end{enumerate}
\end{enumerate}
\end{enumerate}
\end{enumerate}
\end{enumerate}
\end{enumerate}
\end{enumerate}
\end{enumerate}
\end{enumerate}
\end{enumerate}
\end{enumerate}
\end{enumerate}
\end{enumerate}
\end{enumerate}
\end{enumerate}
\end{enumerate}
\end{enumerate}
\end{enumerate}
\end{enumerate}
\end{enumerate}
\end{enumerate}
\end{enumerate}
\end{enumerate}
\end{enumerate}
\end{enumerate}
\end{enumerate}
\end{enumerate}
\end{enumerate}
\end{enumerate}
\end{enumerate}
\end{enumerate}
\end{enumerate}
\end{enumerate}
\end{enumerate}
\end{enumerate}
\end{enumerate}
\end{enumerate}
\end{enumerate}
\end{enumerate}
\end{enumerate}
\end{enumerate}
\end{enumerate}
\end{enumerate}
\end{enumerate}
\end{enumerate}
\end{enumerate}
\end{enumerate}
\end{enumerate}
\end{enumerate}
\end{enumerate}
\end{enumerate}
\end{enumerate}
\end{enumerate}
\end{enumerate}
\end{enumerate}
\end{enumerate}
\end{enumerate}
\end{enumerate}
\end{enumerate}
\end{enumerate}
\end{enumerate}
\end{enumerate}
\end{enumerate}
\end{enumerate}
\end{enumerate}
\end{enumerate}
\end{enumerate}
\end{enumerate}
\end{enumerate}
\end{enumerate}
\end{enumerate}
\end{enumerate}
\end{enumerate}
\end{enumerate}
\end{enumerate}
\end{enumerate}
\end{enumerate}

\item
and upper roman literals
\end{enumerate}
\begin{enumerate}[label=\alph*.]
\item
enumerator enclosed in parenthesis
\end{enumerate}

Vim reStructured \TeX{}t is taking numeration of first element as numeration of whole
list. This code:

\begin{ttfamily}\begin{flushleft}
\mbox{~a.~alpha~list}\\
\mbox{~2.~decimal~list}\\
\end{flushleft}\end{ttfamily}

Will be rendered as:

 \begin{enumerate}[label=\alph*.]
\item
alpha list

\item
decimal list
\end{enumerate}

Two lists of the same type must have separator between them. In other
cause they will be rendered as one:

\begin{ttfamily}\begin{flushleft}
\mbox{~1.~List1~Elem1}\\
\mbox{~2.~List1~Elem2}\\
\mbox{}\\
\mbox{~1.~List2~Elem1}\\
\mbox{~2.~List2~Elem2}\\
\end{flushleft}\end{ttfamily}

Results in:

\begin{enumerate}[label=\arabic*.]
\item
List1 Elem1

\item
List1 Elem2

\item
List2 Elem1

\item
List2 Elem2
\end{enumerate}

Placing between them empty \href{\#lcomment}{comment} is enough. Anyway, short
description is always good thing.

List does not have to start from 1, a or A. Example:

\begin{ttfamily}\begin{flushleft}
\mbox{~5.~First~element~of~list}\\
\mbox{~\#.~Second~element}\\
\end{flushleft}\end{ttfamily}

Will become:

\begin{enumerate}[label=\arabic*.,start=5]
\item
First element of list

\item
Second element
\end{enumerate}

And:

\begin{ttfamily}\begin{flushleft}
\mbox{~h.~lower~alpha~list}\\
\mbox{~\#.~also~don't~have~to~start~from~a}\\
\end{flushleft}\end{ttfamily}

Results in:

\begin{enumerate}[label=\alph*.,start=8]
\item
lower alpha list

\item
also don't have to start from a
\end{enumerate}

Also roman numerals can begin not only with i/I. Note however they have to
begin with number containing more than one letter (xl, iii, CV), in other case
they will be treated as alpha lists. Also each list beginning with i/I will be
treated as roman, never as alpha which begins at 9\sups{th} letter of Latin
alphabet.

\begin{center}
\fbox{\begin{minipage}{0.8\textwidth}
\textbf{\sffamily\large Warning}
\vspace{2mm}

In \LaTeX{} export starting of lists not from 1/a/A/i/I requires
\href{http://www.ctan.org/TeX-archive/macros/\LaTeX{}/contrib/enumitem}{enumitem} package.
\end{minipage}}
\end{center}
\hypertarget{lauto-numerated-lists}{}
\subsubsection{Auto numerated lists}

Lists can be auto numerated. Begin list item with \texttt{\#}:

\begin{ttfamily}\begin{flushleft}
\mbox{~\#.~First~item~of~list}\\
\mbox{~\#.~Second~item~of~list}\\
\end{flushleft}\end{ttfamily}

All lists created with \texttt{\#} will be displayed as numerated lists. Result:

\begin{enumerate}[label=\arabic*.]
\item
First item of list

\item
Second item of list
\end{enumerate}
\hypertarget{lbulleted-lists}{}
\subsubsection{Bulleted lists}

Just like enumerated lists, start the line off with a bullet point
character -- either "-", "+" or "*":

\begin{ttfamily}\begin{flushleft}
\mbox{~*~a~bullet~using~"*"}\\
\mbox{}\\
\mbox{~-~list~using~"-"}\\
\mbox{}\\
\mbox{~+~yet~another~list}\\
\end{flushleft}\end{ttfamily}

Results in:

 \begin{itemize}
\item
a bullet using "*"
 \end{itemize}
 \begin{itemize}
\item
list using "-"
 \end{itemize}
 \begin{itemize}
\item
yet another list
\end{itemize}

These elements are connected (HTML only). \texttt{*} bulleted list always will be
\texttt{circle}, \texttt{-} will be \texttt{disc} and \texttt{+} will be \texttt{square}.

In UTF-8 it is possible to use unicode BULLET characters:

\begin{itemize}
\item
$\backslash$u2022 BULLET

\item
$\backslash$u2023 TRIANGULAR BULLET

\item
$\backslash$u2043 HYPHEN BULLET

\item
$\backslash$u204C BLACK LEFTWARDS BULLET

\item
$\backslash$u204D BLACK RIGHTWARDS BULLET

\item
$\backslash$u2219 BULLET OPERATOR

\item
$\backslash$u25D8 INVERSE BULLET

\item
$\backslash$u25E6 WHITE BULLET

\item
$\backslash$u2619 REVERSED ROTATED FLORAL HEART BULLET

\item
$\backslash$u2765 ROTATED HEAVY BLACK HEART BULLET

\item
$\backslash$u2767 ROTATED FLORAL HEART BULLET

\item
$\backslash$u29BE CIRCLED WHITE BULLET

\item
$\backslash$u29BF CIRCLED BULLET
\end{itemize}

To make nicely looking \TeX{}t documents, however all of them will be treated in
output as \texttt{-}.

\hypertarget{lembedding-of-lists}{}
\subsubsection{Embedding of lists}

Enumerated and bulleted lists can contain many elements and can be
nested. This code will be rendered:

\begin{ttfamily}\begin{flushleft}
\mbox{~1.~This~is~description~how~to~make~lists~embeddable}\\
\mbox{~~~~one~into~other.}\\
\mbox{}\\
\mbox{~~~~~~~~-~start~one~list}\\
\mbox{~~~~~~~~-~insert~blank~line~and~bigger~indentation}\\
\mbox{~~~~~~~~-~start~another~list}\\
\mbox{~~~~~~~~-~insert~blank~line~before~next~element}\\
\mbox{}\\
\mbox{~2.~It~is~possible~to~embed~paragraphs~into~list~(and~blockquotes)}\\
\mbox{~~~~also.}\\
\mbox{}\\
\mbox{~~~~Paragraphs~and~blockquotes~have~to~be~separated~by~blank~line~and}\\
\mbox{~~~~start~where~list~begins:~indentation~of~list~"leader"~plus}\\
\mbox{~~~~"leader",~punctuation~sign~and~space.}\\
\mbox{}\\
\mbox{~~~~~~~~~That~is~memorable~quote.}\\
\mbox{}\\
\mbox{~~~~~Those~features~are~not~implemented~for~other~types~of~elements.}\\
\mbox{~~~~~Only:~plain~paragraphs,~blockquotes,~ordered~lists,~bulleted}\\
\mbox{~~~~~lists.}\\
\mbox{}\\
\mbox{~~~~~Indentation~level~is~*very*~important.~One~space~can~break}\\
\mbox{~~~~~things.}\\
\end{flushleft}\end{ttfamily}

Results in:

\begin{enumerate}[label=\arabic*.]
\item
This is description how to make lists embeddable
one into other.

 \begin{itemize}
\item
start one list

\item
insert blank line and bigger indentation

\item
start another list

\item
insert blank line before next element
\end{itemize}

\item
It is possible to embed paragraphs into list (and blockquotes)
also.

 Paragraphs and blockquotes have to be separated by blank line and
 start where list begins: indentation of list "leader" plus
 "leader", punctuation sign and space.

 \begin{quotation}
 That is memorable quote.
 \end{quotation}

 Those features are not implemented for other types of elements. Only:
 plain paragraphs, blockquotes, ordered lists, bulleted lists.

 Indentation level is \emph{very} important. One space can break things.
\end{enumerate}
\hypertarget{ldefinition-lists}{}
\subsubsection{Definition lists}

Unlike the other two, the definition lists consist of a term, and
the definition of that term. The format of a definition list is:

\begin{ttfamily}\begin{flushleft}
\mbox{~what}\\
\mbox{~~~~~Definition~lists~associate~a~term~with~a~definition.}\\
\mbox{}\\
\mbox{~*how*}\\
\mbox{~~~~~The~term~is~a~one-line~phrase,~and~the~definition~are~body~elements,}\\
\mbox{~~~~~indented~relative~to~the~term.~~Blank~lines~are~not~allowed~between}\\
\mbox{~~~~~term~and~first~line~of~definition.}\\
\mbox{}\\
\mbox{~list}\\
\mbox{~~~~~1.~First~element~of~definition}\\
\mbox{~~~~~2.~don't~have~to~be}\\
\mbox{~~~~~3.~plain~paragraph.}\\
\end{flushleft}\end{ttfamily}

Results in:

\begin{deflist}{iii}

\item[what]

Definition lists associate a term with a definition.

\item[\emph{how}]

The term is a one-line phrase, and the definition are body elements,
indented relative to the term. Blank lines are not allowed between term
and first line of definition.

\item[list]

\begin{enumerate}[label=\arabic*.]
\item
First element of definition

\item
don't have to be

\item
plain paragraph.
\end{enumerate}
\end{deflist}
\hypertarget{lfield-list}{}
\subsection{Field list}

Special kind of list designed for headers of files or highlighting
important information. Paragraph in form:

\begin{ttfamily}\begin{flushleft}
\mbox{~:Author:~Mikolaj~Machowski}\\
\mbox{~:Something:~Somewhere}\\
\end{flushleft}\end{ttfamily}

Results in:

\begin{deflist}{Keywords:}

\item[Author:] Mikolaj Machowski

\item[Something:] Somewhere
\end{deflist}

Vim recognizes some names of field list as special and places them in
meta info of created documents:

\begin{itemize}
\item
Author

\item
Title

\item
Date

\item
Subject

\item
Keywords
\end{itemize}

By default date is displayed in \LaTeX{} documents. User can suppress it by using
keyword NONE:

\begin{ttfamily}\begin{flushleft}
\mbox{~:Date:~NONE}\\
\end{flushleft}\end{ttfamily}

This line will be removed from output and there will be no info about date in
exported document.

\hypertarget{loption-list}{}
\subsection{Option list}

Vim reStructured \TeX{}t recognizes also special type of lists: option lists. It is
designed for listing of command options and Vim commands.

It is possible to create list from various types of command line
options:

\begin{ttfamily}\begin{flushleft}
\mbox{~~-a~~~~~~~~~~~~~~~~~~Short~form~of~an~option}\\
\mbox{~~--all-name~~~~~~~~~~Long~form~of~an~option}\\
\mbox{~~-a,~--all-name~~~~~~Combined~listing~of~option}\\
\mbox{~~+option~~~~~~~~~~~~~Old~style~options}\\
\mbox{~~/VMS~~~~~~~~~~~~~~~~VMS~and~DOS~style~options}\\
\mbox{~~:Vim~command~~~~~~~~And~Vim~command~--~not~available~in~reST}\\
\mbox{~~--multi-struct~~~~~~It~is~possible~to~use~structure~elements~in}\\
\mbox{~~~~~~~~~~~~~~~~~~~~~~option~lists::}\\
\mbox{}\\
\mbox{~~~~~~~~~~~~~~~~~~~~~~~~~~~To~present~examples~of~use}\\
\mbox{}\\
\mbox{~~~~~~~~~~~~~~~~~~~~~~1.~Or~possible}\\
\mbox{~~~~~~~~~~~~~~~~~~~~~~2.~uses}\\
\mbox{~~~~~~~~~~~~~~~~~~~~~~3.~Or~any~other~structure~element.}\\
\mbox{}\\
\mbox{~--in-case-of-very-long-option}\\
\mbox{~~~~~~~~~~~~~~~~~~~~~~definition~may~begin~in~next~line,~no~trailing~space}\\
\mbox{~~~~~~~~~~~~~~~~~~~~~~after~option~name~allowed.}\\
\end{flushleft}\end{ttfamily}

Option (command) have to be separated from description by at least two
spaces. Above example results in:

\begin{optlist}{longoptionslist}

\item[-a]

 Short form of an option

\item[--all-name]

 Long form of an option

\item[-a, --all-name]

 Combined listing of option

\item[+option]

 Old style options

\item[/VMS]

 VMS and DOS style options

\item[:Vim command]

 And Vim command -- not available in reST

\item[--multi-struct]

 It is possible to use structure elements in
 option lists:
\begin{ttfamily}\begin{flushleft}
\mbox{~To~present~examples~of~use}\\
\end{flushleft}\end{ttfamily}

 \begin{enumerate}[label=\arabic*.]
\item
Or possible

\item
uses

\item
Or any other structure element.
\end{enumerate}
\end{optlist}
\begin{optlist}{longoptionslist}

\item[--in-case-of-very-long-option]

 definition may begin in next line, no trailing space
 after option name allowed.
\end{optlist}
\hypertarget{lline-blocks}{}
\subsection{Line blocks}

Useful for presenting poetry or some preformatted \TeX{}t but not in monospace
font like addresses:

\begin{ttfamily}\begin{flushleft}
\mbox{~|~This~is~*first*~line}\\
\mbox{~|~~~~~~This~is~indented~**second**~line}\\
\mbox{~|~~~This~is~indented~slightly~less~then~second}\\
\mbox{~~~~~~~~~~~~~line,~third~(long)~line}\\
\mbox{~|~And~last,~:small:`fourth`~line}\\
\end{flushleft}\end{ttfamily}

Will result in:

\begin{verse}
~This is \emph{first} line\\
~~~~~~This is indented \textbf{second} line\\
~~~This is indented slightly less then second
 line, third (long) line\\
~And last, {\small fourth} line
\end{verse}

As you see, you can use markup inside of line block paragraphs. Line without
"$|$ " starting sequence will be treated as continuation of previous line.

To use $|$ at the beginning of paragraph but without intention of line block
creation, escape it with backslash.

\begin{center}
\fbox{\begin{minipage}{0.8\textwidth}
\textbf{\sffamily\large Tip}
\vspace{2mm}

Indentation provided by output writers is visually much smaller than
presented in monospace font. Make correction for that.
\end{minipage}}
\end{center}

\hypertarget{ldouble-colon}{}

\hypertarget{lpreformatting}{}
\subsection{Preformatting}

To just include a chunk of preformatted, \TeX{}t, finish the prior
paragraph with "\texttt{::}". The preformatted block is finished when the
\TeX{}t falls back to the same indentation level as a paragraph prior to
the preformatted block. For example:

\begin{ttfamily}\begin{flushleft}
\mbox{~An~example::}\\
\mbox{}\\
\mbox{~~~~~~~Whitespace,~newlines,~blank~lines,~and~all~kinds~of~markup}\\
\mbox{~~~~~~~~~(like~*this*~or~$\backslash$this)~is~preserved~by~literal~blocks.}\\
\mbox{~~~~~Lookie~here,~I've~dropped~an~indentation~level}\\
\mbox{~~~~~(but~not~far~enough)}\\
\mbox{}\\
\mbox{~no~more~example}\\
\end{flushleft}\end{ttfamily}

Results in:

An example:

\begin{ttfamily}\begin{flushleft}
\mbox{~~~Whitespace,~newlines,~blank~lines,~and~all~kinds~of~markup}\\
\mbox{~~~~~(like~*this*~or~$\backslash$this)~is~preserved~by~literal~blocks.}\\
\mbox{~Lookie~here,~I've~dropped~an~indentation~level}\\
\mbox{~(but~not~far~enough)}\\
\end{flushleft}\end{ttfamily}

no more example

\hypertarget{lempty-double-colon}{}

Note that if a paragraph consists only of "\texttt{::}", then it's removed
from the output:

\begin{ttfamily}\begin{flushleft}
\mbox{~::}\\
\mbox{}\\
\mbox{~~~~~This~is~preformatted~\TeX{}t,~and~the}\\
\mbox{~~~~~last~"::"~paragraph~is~removed}\\
\end{flushleft}\end{ttfamily}

Results in:

\begin{ttfamily}\begin{flushleft}
\mbox{~This~is~preformatted~\TeX{}t,~and~the}\\
\mbox{~last~"::"~paragraph~is~removed}\\
\end{flushleft}\end{ttfamily}

\hypertarget{lquoted-literal-blocks}{}
\subsubsection{Quoted literal blocks}

Quoted literal blocks are unindented blocks of \TeX{}t where each line begins
with the same character:

\begin{ttfamily}\begin{flushleft}
\mbox{~!~"~\#~\$~\%~\&~'~(~)~*~+~,~-~.~/~:~;~<~=~>~?~@~[~]~\^{}~\_~`~\{~|~\}~\~{}}\\
\end{flushleft}\end{ttfamily}

\textbf{And} previous paragraph ends with \texttt{::}.

Blank line ends quoted literal block. Quoting characters are preserved.
Example:

\begin{ttfamily}\begin{flushleft}
\mbox{~You~wrote::}\\
\mbox{}\\
\mbox{~>>~Thanks~for~your~work}\\
\mbox{~>}\\
\mbox{~>~Glad~you~appreciate~it}\\
\mbox{}\\
\mbox{~Ha!}\\
\end{flushleft}\end{ttfamily}

Results in:

You wrote:

\begin{ttfamily}\begin{flushleft}
\mbox{~>>~Thanks~for~your~work}\\
\mbox{~>}\\
\mbox{~>~Glad~you~appreciate~it}\\
\end{flushleft}\end{ttfamily}

Ha!

\hypertarget{ldoctest}{}
\subsection{Doctest}

Special case of preformatted \TeX{}t are doctest blocks. First line have to begin
with "$>$$>$$>$" and can contain only one paragraph of \TeX{}t (without blank lines):

\begin{ttfamily}\begin{flushleft}
\mbox{~>>>~print~'Python-specific~usage~examples;~begun~with~">>>"'}\\
\mbox{~Python-specific~usage~examples;~begun~with~">>>"}\\
\mbox{~>>>~print~'(cut~and~pasted~from~interactive~Python~sessions)'}\\
\mbox{~(cut~and~pasted~from~interactive~Python~sessions)}\\
\end{flushleft}\end{ttfamily}

\hypertarget{lsections}{}
\subsection{Sections}

To break longer \TeX{}t up into sections, you use \textbf{section headers}. These are
a single line of \TeX{}t (one or more words) with adornment: an underline, in
dashes "\texttt{-----}", equals "\texttt{======}", tildes "\texttt{\~{}\~{}\~{}\~{}\~{}\~{}}" or any of the
non-alphanumeric characters \texttt{= - \~{} \^{} ` * + \# } that you feel comfortable
with (full list of chars is in \href{\#lquoted-literal-blocks}{quoted literal blocks} section). The
underline must be at least as long as the title \TeX{}t. Be consistent, since
all sections marked with the same adornment style are deemed to be at the same
level:

\begin{ttfamily}\begin{flushleft}
\mbox{~Chapter~1~Title}\\
\mbox{~===============}\\
\mbox{}\\
\mbox{~Section~1.1~Title}\\
\mbox{~-----------------}\\
\mbox{}\\
\mbox{~Subsection~1.1.1~Title}\\
\mbox{~\~{}\~{}\~{}\~{}\~{}\~{}\~{}\~{}\~{}\~{}\~{}\~{}\~{}\~{}\~{}\~{}\~{}\~{}\~{}\~{}\~{}\~{}}\\
\mbox{}\\
\mbox{~Section~1.2~Title}\\
\mbox{~-----------------}\\
\mbox{}\\
\mbox{~Chapter~2~Title}\\
\mbox{~===============}\\
\end{flushleft}\end{ttfamily}

To make some section titles more outstanding you can use double style headers,
with adornments below and \emph{above} of title. These special lines \textbf{must} be
equal, both in characters and length. However, these two titles:

\begin{ttfamily}\begin{flushleft}
\mbox{~=================}\\
\mbox{~Document~title}\\
\mbox{~=================}\\
\mbox{}\\
\mbox{~Section~title}\\
\mbox{~=============}\\
\end{flushleft}\end{ttfamily}

Will be treated as two different levels.

In HTML export sections will be numbered thanks to \texttt{content} property.
Alas, only small number of WWW browsers are supporting this feature
(\href{http://www.kde.org}{Konqueror}, \href{http://www.opera.com}{Opera}, Firefox 1.5).

Section headers don't have to be separated with blank line from next paragraph
but it is recommended. Simple paragraphs not separated from section header
will be treated as \href{\#lsubtitles}{subtitles}, rest will be treated normally, only directives
and special markup explicit blocks are forbidden.

\hypertarget{lsubtitles}{}
\subsubsection{Subtitles}

It is possible to provide subtitles for section headers. It should be one,
short paragraph placed directly under ornament which will be rendered slightly
bigger than normal \TeX{}t. Example:

\begin{ttfamily}\begin{flushleft}
\mbox{~Directives}\\
\mbox{~----------}\\
\mbox{~Or~how~to~place~special~elements~in~\TeX{}t}\\
\end{flushleft}\end{ttfamily}

Check rendering of \href{\#ldirectives}{Directives} section header.

\hypertarget{llinks}{}
\subsection{Links}

Links are important part of modern document. Vim reStructured \TeX{}t allows to create
external and internal links. All names declarations are case
insensitive. It means both examples will be working:

\begin{ttfamily}\begin{flushleft}
\mbox{~start\_}\\
\mbox{}\\
\mbox{~\_start:~http://www.vim.org}\\
\mbox{}\\
\mbox{~`Starting~point`\_}\\
\mbox{}\\
\mbox{~..~\_starting~point:~http://www.vim.org}\\
\end{flushleft}\end{ttfamily}

Jump to \href{\#ltables}{tables} (which is section with title "Tables").

\hypertarget{lstarting-point}{}
\subsubsection{Starting point}

Starting point looks like this:

\begin{ttfamily}\begin{flushleft}
\mbox{~We~explained~`starting~point`\_~somewhere~else}\\
\mbox{}\\
\mbox{~The~same~for~start\_}\\
\end{flushleft}\end{ttfamily}

Note: when start is single entity made from \texttt{[:alnum:]}, \texttt{.}, \texttt{-},
\texttt{\_} characters it may not be enclosed in backticks, also if word
is constructed from \texttt{iskeyword} characters.

\hypertarget{lexternal-links}{}
\subsubsection{External links}

Definition of external target:

\begin{ttfamily}\begin{flushleft}
\mbox{~..~\_starting~point:~http://www.vim.org}\\
\mbox{}\\
\mbox{~..~\_start:~http://skawina.eu.org/mikolaj}\\
\end{flushleft}\end{ttfamily}

Note: lack of backticks around titles, even when there is more than one
word. Links can be split for several lines:

\begin{ttfamily}\begin{flushleft}
\mbox{~..~\_very,~very~long~link~description:}\\
\mbox{~~~~~~~~http://this.is.address.com/of/this/description}\\
\end{flushleft}\end{ttfamily}

\hypertarget{linternal-links}{}
\subsubsection{Internal links}

Definition of internal target can be done in two ways.

First is to put definition in \TeX{}t:

\begin{ttfamily}\begin{flushleft}
\mbox{~some~\TeX{}t~about~\_`starting~point`~explaining~this~term}\\
\end{flushleft}\end{ttfamily}

Backticks are obligatory.

Second way is anonymous target:

\begin{ttfamily}\begin{flushleft}
\mbox{~..~\_starting~point:}\\
\end{flushleft}\end{ttfamily}

Very similar to external target just pointing nowhere.

\hypertarget{lstandalone-links}{}
\subsubsection{Standalone links}

Links can be put directly into \TeX{}t when written explicitly in \TeX{}t:

\begin{ttfamily}\begin{flushleft}
\mbox{~This~link~to~http://skawina.eu.org/mikolaj~page~by}\\
\mbox{~mailto:mikmach@wp.pl.}\\
\end{flushleft}\end{ttfamily}

Results in:

This link to \href{http://skawina.eu.org/mikolaj}{http://skawina.eu.org/mikolaj} page by \href{mailto:mikmach@wp.pl}{mikmach@wp.pl}.

Supported protocols: http, https, ftp, mailto.

\hypertarget{lanonymous-hyperlinks}{}
\subsubsection{Anonymous hyperlinks}

Definitions of links are boring. For creation of fast links use anonymous
hyperlinks. Example:

\begin{ttfamily}\begin{flushleft}
\mbox{~This~is~link\_\_~to~external~document.~I~don't~have~time~to~write\_\_~full}\\
\mbox{~definition.}\\
\mbox{}\\
\mbox{~\_\_~http://link.to.some.external.doc}\\
\mbox{}\\
\mbox{~..~\_\_:~http://second.link.to.external.doc}\\
\end{flushleft}\end{ttfamily}

As you can see order of links and definitions is important. Should be used
only for fast and dirty work.

\hypertarget{lindirect-links}{}
\subsubsection{Indirect links}

Links definitions can be starting points defined elsewhere. Example:

\begin{ttfamily}\begin{flushleft}
\mbox{~..~\_my~wonderful~page:~start\_}\\
\mbox{}\\
\mbox{~..~\_start:~http://skawina.eu.org/mikolaj}\\
\end{flushleft}\end{ttfamily}

Should point to \href{http://skawina.eu.org/mikolaj}{my wonderful page}.

This can be also used in anonymous links:

\begin{ttfamily}\begin{flushleft}
\mbox{~\_\_~start\_}\\
\end{flushleft}\end{ttfamily}

User can even create multi element chains:

\begin{ttfamily}\begin{flushleft}
\mbox{~..~\_first~elem:~secondelem\_}\\
\mbox{}\\
\mbox{~..~\_secondelem:~thirdelem\_}\\
\mbox{}\\
\mbox{~..~\_thirdelem:~http://skawina.eu.org/mikolaj}\\
\end{flushleft}\end{ttfamily}

Here is \href{http://skawina.eu.org/mikolaj}{first elem} link.

\hypertarget{lembedded-uris}{}
\subsubsection{Embedded URIs}

A hyperlink reference may directly embed a target URI inline, within
angle brackets ("$<$...$>$") as follows:

\begin{ttfamily}\begin{flushleft}
\mbox{~See~the~`Vim-online~page~<http://www.vim.org>`\_~for~info.}\\
\end{flushleft}\end{ttfamily}

This is exactly equivalent to:

\begin{ttfamily}\begin{flushleft}
\mbox{~See~the~`Vim-online~page`\_~for~info.}\\
\mbox{}\\
\mbox{~..~\_Vim-online~page:~http://www.vim.org}\\
\end{flushleft}\end{ttfamily}

The bracketed URI must be preceded by whitespace and be the last \TeX{}t
before the end string. With a single trailing underscore, the
reference is named and the same target URI may be referred to again.

With two trailing underscores, the reference and target are both
anonymous, and the target cannot be referred to again. These are
"one-off" hyperlinks in form:

\begin{ttfamily}\begin{flushleft}
\mbox{~This~is~`embedded~URI~<examples.html>`\_\_}\\
\mbox{~This~is~`John~Smith's~mail~<mailto:john@smith.info>`\_\_}\\
\end{flushleft}\end{ttfamily}

Note double underscore at the end and declaration of \texttt{mailto:} with e-mail.

There is also possible ultimate compact form:

\begin{ttfamily}\begin{flushleft}
\mbox{~This~is~link~to~`<vst-changelog.html>`\_\_.}\\
\end{flushleft}\end{ttfamily}

Results in:

This is link to \href{vst-changelog.html}{vst-changelog.html}.

\hypertarget{lreplacement-link-combo}{}
\subsubsection{Replacement-link combo}

It is often boring to write long link \TeX{}ts. Shortening of them is very handy.
Vim reStructured \TeX{}t can do that with:

\begin{ttfamily}\begin{flushleft}
\mbox{~This~is~|vrest|\_.}\\
\mbox{}\\
\mbox{~..~|vrest|~replace::~reST~implementation~for~Vim,~**the~best**~editor~of~Earth}\\
\mbox{}\\
\mbox{~..~\_vrest:~http://skawina.eu.org/mikolaj/vst.html}\\
\end{flushleft}\end{ttfamily}

Results in:

This is \href{http://skawina.eu.org/mikolaj/vst.html}{reST implementation for Vim, \textbf{the best} editor of Earth}.

This also a way to use inline markup inside of links.

\hypertarget{ltransitions}{}
\subsection{Transitions}

It is a nice touch to separate some paragraphs and parts of \TeX{}t
visually. In some old-fashioned books it is done with small graphics, in
newer with eg. row of asterisks \texttt{* * *}.

In Vim reStructured \TeX{}t you can do this with line of letters, preferred are characters
used for \href{\#lsections}{sections} underscoring:

\begin{ttfamily}\begin{flushleft}
\mbox{~================================================}\\
\mbox{}\\
\mbox{~------------------------------------------------}\\
\mbox{}\\
\mbox{~\~{}\~{}\~{}\~{}\~{}\~{}\~{}\~{}\~{}\~{}\~{}\~{}\~{}\~{}\~{}\~{}\~{}\~{}\~{}\~{}\~{}\~{}\~{}\~{}\~{}\~{}\~{}\~{}\~{}\~{}\~{}\~{}\~{}\~{}\~{}\~{}\~{}\~{}\~{}\~{}\~{}\~{}\~{}\~{}\~{}\~{}\~{}\~{}}\\
\mbox{}\\
\mbox{~************************************************}\\
\end{flushleft}\end{ttfamily}

etc. It have to be separated from other elements with blank lines. In
exported file they will look like straight line:

\transition

\hypertarget{lattribution}{}
\subsection{Attribution}

When quoting \TeX{}t it is nice to add mention about author of quote. Special
element of \TeX{}t looks like:

\begin{ttfamily}\begin{flushleft}
\mbox{~This~is~memorable~quote.}\\
\mbox{}\\
\mbox{~~~~~~~~~--~John~Smith,~Esq.}\\
\end{flushleft}\end{ttfamily}

Results in:

 \begin{quotation}
 This is memorable quote.

 \begin{quotation}
 \attribution{-- John Smith, Esq.
 }
 \end{quotation}
 \end{quotation}

Some things which may not be visible:

 \begin{itemize}
\item
Must be last paragraph of block quote

\item
Must begin with '--' or '---' and space
\end{itemize}
\hypertarget{ltables}{}
\subsection{Tables}

Vim reStructured \TeX{}t provides support for two types of tables. With border:

\begin{ttfamily}\begin{flushleft}
\mbox{~+---------------------+----------------+------------------+}\\
\mbox{~|~Cells~are~~~~~~~~~~~|~by~bar~with~~~~|~<Space>|<Space>~~|}\\
\mbox{~|~separated~~~~~~~~~~~|~spaces~around~~|~~~~~~~~~~~~~~~~~~|}\\
\mbox{~+---------------------+----------------+------------------+}\\
\mbox{~|~*You*~may~use~~~~~~~|~markup,~~~~~~~~|~included.~~~~~~~~|}\\
\mbox{~|~**inline**~~~~~~~~~~|~links\_~~~~~~~~~|~~~~~~~~~~~~~~~~~~|}\\
\mbox{~+---------------------+----------------+------------------+}\\
\mbox{~|~You~can~use~various~|~-~like~lists~~~|~|VST|~will~~~~~~~|}\\
\mbox{~|~types~of~structure~~|~~~~~~~~~~~~~~~~|~interpret~them.~~|}\\
\mbox{~|~elements::~~~~~~~~~~|~~~~+~various~~~|~~~~~~~~~~~~~~~~~~|}\\
\mbox{~|~~~~~~~~~~~~~~~~~~~~~|~~~~~~~~~~~~~~~~|~Even~paragraphs.~|}\\
\mbox{~|~~Welcome~to~world~~~|~-~embedded~~~~~|~~~~~~~~~~~~~~~~~~|}\\
\mbox{~|~~of~preformatted~~~~|~-~one~into~~~~~|~1.~And~~~~~~~~~~~|}\\
\mbox{~|~~\TeX{}t.~~~~~~~~~~~~~~|~~~other~~~~~~~~|~2.~not~~~~~~~~~~~|}\\
\mbox{~|~~~~~~~~~~~~~~~~~~~~~|~~~~~~~~~~~~~~~~|~3.~only~~~~~~~~~~|}\\
\mbox{~+---------------------+----------------+------------------+}\\
\mbox{~|~\TeX{}t~may~span~across~several~~~~~~~~~|~~~~~~~~~~~~~~~~~~|}\\
\mbox{~|~columns.~Cell~can~be~also~empty~->~~~|~~~~~~~~~~~~~~~~~~|}\\
\mbox{~+---------------------+----------------+------------------+}\\
\end{flushleft}\end{ttfamily}

This is result of table:

\setlongtables
\begin{center}
\begin{longtable}[c]{|p{0.34\\textwidth}|p{0.26\\textwidth}|p{0.29\\textwidth}|}\hline

Cells are 
separated
&
by bar with 
spaces around
&
$<$Space$>$$|$$<$Space$>$
\\ \hline
\emph{You} may use 
\textbf{inline}
&
markup, 
\href{\#llinks}{links}
&
included.
\\ \hline
You can use various
types of structure 
elements:

\begin{ttfamily}\begin{flushleft}
\mbox{~Welcome~to~world~~}\\
\mbox{~of~preformatted~~~}\\
\mbox{~\TeX{}t.}\\
\end{flushleft}\end{ttfamily}
&
\begin{itemize}
\item
like lists

 \begin{itemize}
\item
various
\end{itemize}

\item
embedded

\item
one into 
other
\end{itemize}
&
Vim reStructured \TeX{}t will 
interpret them.

Even paragraphs.

\begin{enumerate}[label=\arabic*.]
\item
And

\item
not

\item
only
\end{enumerate}

 \\ \hline
\multicolumn{2}{|p{0.60\textwidth}|}{\TeX{}t may span across several 
columns. Cell can be also empty -$>$
}
&
\\ \hline\end{longtable}
\end{center}

For tables containing bigger chunks of structured \TeX{}t it may be better
to use border less tables. They are looking almost the same as regular
tables with exception of first line which is created with from equal
sign:

\begin{ttfamily}\begin{flushleft}
\mbox{~+======================================================================+}\\
\mbox{~|~~~~~This~is~converted~fragment~of~ChangeLog\_~~~~~~~~~~~~~~~~~~~~~~~~~|}\\
\mbox{~+============+=========================================================+}\\
\mbox{~|~5~Apr~2005~|~-~FIX:~[HTML]~properly~indent~preformatted~~~~~~~~~~~~~~|}\\
\mbox{~|~~~~~~~~~~~~|~~~\TeX{}t~when~first~line~has~bigger~~~~~~~~~~~~~~~~~~~~~~~|}\\
\mbox{~|~~~~~~~~~~~~|~~~indentation~than~next~ones~~~~~~~~~~~~~~~~~~~~~~~~~~~~|}\\
\mbox{~|~~~~~~~~~~~~|~-~CHG:~[\LaTeX{}]~improve~displaying~~~~~~~~~~~~~~~~~~~~~~~|}\\
\mbox{~|~~~~~~~~~~~~|~~~of~field~lists~~~~~~~~~~~~~~~~~~~~~~~~~~~~~~~~~~~~~~~~|}\\
\mbox{~+------------+---------------------------------------------------------+}\\
\mbox{~|~6~Apr~2005~|~-~ADD:~Raw\LaTeX{}~~directive~~~~~~~~~~~~~~~~~~~~~~~~~~~~~~|}\\
\mbox{~|~~~~~~~~~~~~|~-~ADD:~[HTML]~use~counters~in~CSS~for~~~~~~~~~~~~~~~~~~~|}\\
\mbox{~|~~~~~~~~~~~~|~~~numbering~of~`table~of~contents`\_~~~~~~~~~~~~~~~~~~~~~|}\\
\mbox{~|~~~~~~~~~~~~|~~~and~sections\_~in~\TeX{}t.~At~the~moment~~~~~~~~~~~~~~~~~~|}\\
\mbox{~|~~~~~~~~~~~~|~~~this~numbering~can~be~seen~only~in~~~~~~~~~~~~~~~~~~~~|}\\
\mbox{~|~~~~~~~~~~~~|~~~Konqueror\_~3.4~and~Opera\_~ver.~?~~~~~~~~~~~~~~~~~~~~~~|}\\
\mbox{~+------------+---------------------------------------------------------+}\\
\end{flushleft}\end{ttfamily}

Result of above example:

\setlongtables
\begin{center}
\begin{longtable}[c]{p{0.15\\textwidth} p{0.74\\textwidth} }
\multicolumn{2}{p{0.89\textwidth}}{\begin{quotation}
 This is converted fragment of \href{\#lchangelog}{ChangeLog}
 \end{quotation}}
 \\ 
\endhead

5 Apr 2005
&
\begin{itemize}
\item
FIX: [HTML] properly indent preformatted 
\TeX{}t when first line has bigger 
indentation than next ones

\item
CHG: [\LaTeX{}] improve displaying 
of field lists
\end{itemize}

 \\ 

6 Apr 2005
&
\begin{itemize}
\item
ADD: Raw\LaTeX{} directive

\item
ADD: [HTML] use counters in CSS for 
numbering of \href{\#ltable-of-contents}{table of contents} 
and \href{\#lsections}{sections} in \TeX{}t. At the moment 
this numbering can be seen only in 
\href{http://www.kde.org}{Konqueror} 3.4 and \href{http://www.opera.com}{Opera} ver. ?
\end{itemize}

 \\ \end{longtable}
\end{center}

Row separator from \texttt{=} will create head of table. Second such row will
create foot of table (only in HTML export).

\hypertarget{lsimple-tables}{}
\subsection{Simple tables}

Full tables are hard to correct and in most cases not necessary. Simple tables
are much simpler to write and maintain. They have also less features. The most
important difference is lack of support for spanning columns. Fragment of
changelog rewritten as simple table:

\begin{ttfamily}\begin{flushleft}
\mbox{~============~~========================================================}\\
\mbox{~Date~~~~~~~~~~This~is~converted~fragment~of~ChangeLog\_}\\
\mbox{~============~~========================================================}\\
\mbox{~5~Apr~2005~~~~-~FIX:~[HTML]~properly~indent~preformatted~~~~~~~~~~~~~~}\\
\mbox{~~~~~~~~~~~~~~~~~\TeX{}t~when~first~line~has~bigger~~~~~~~~~~~~~~~~~~~~~~~}\\
\mbox{~~~~~~~~~~~~~~~~~indentation~than~next~ones~~~~~~~~~~~~~~~~~~~~~~~~~~~~}\\
\mbox{~~~~~~~~~~~~~~~-~CHG:~[\LaTeX{}]~improve~displaying~~~~~~~~~~~~~~~~~~~~~~~}\\
\mbox{~~~~~~~~~~~~~~~~~of~field~lists~~~~~~~~~~~~~~~~~~~~~~~~~~~~~~~~~~~~~~~~}\\
\mbox{~6~Apr~2005~~~~-~ADD:~Raw\LaTeX{}~~directive~~~~~~~~~~~~~~~~~~~~~~~~~~~~~~}\\
\mbox{~~~~~~~~~~~~~~~-~ADD:~[HTML]~use~counters~in~CSS~for~~~~~~~~~~~~~~~~~~~}\\
\mbox{~~~~~~~~~~~~~~~~~numbering~of~`table~of~contents`\_~~~~~~~~~~~~~~~~~~~~~}\\
\mbox{~~~~~~~~~~~~~~~~~and~sections\_~in~\TeX{}t.~At~the~moment~~~~~~~~~~~~~~~~~~}\\
\mbox{~~~~~~~~~~~~~~~~~this~numbering~can~be~seen~only~in~~~~~~~~~~~~~~~~~~~~}\\
\mbox{~~~~~~~~~~~~~~~~~Konqueror\_~3.4~and~Opera\_~ver.~?~~~~~~~~~~~~~~~~~~~~~~}\\
\mbox{~============~~========================================================}\\
\end{flushleft}\end{ttfamily}

Results in:

\setlongtables
\begin{center}
\begin{longtable}[c]{|p{0.17\\textwidth}|p{0.72\\textwidth}|}\hline

Date
&
This is converted fragment of \href{\#lchangelog}{ChangeLog}

 \\ \hline
\endhead

5 Apr 2005
&
\begin{itemize}
\item
FIX: [HTML] properly indent preformatted
\TeX{}t when first line has bigger
indentation than next ones

\item
CHG: [\LaTeX{}] improve displaying
of field lists
\end{itemize}
 \\ \hline

6 Apr 2005
&
\begin{itemize}
\item
ADD: Raw\LaTeX{} directive

\item
ADD: [HTML] use counters in CSS for
numbering of \href{\#ltable-of-contents}{table of contents}
and \href{\#lsections}{sections} in \TeX{}t. At the moment
this numbering can be seen only in
\href{http://www.kde.org}{Konqueror} 3.4 and \href{http://www.opera.com}{Opera} ver. ?
\end{itemize}
 \\ \hline
\end{longtable}
\end{center}

Another nice example is output of \texttt{cal} program with slight modifications:

\begin{ttfamily}\begin{flushleft}
\mbox{~==~==~==~==~==~==~==}\\
\mbox{~su~mo~tu~we~th~fr~sa~}\\
\mbox{~==~==~==~==~==~==~==}\\
\mbox{~~~~~~~~~~~~~~~~~~~1}\\
\mbox{~2~~3~~4~~5~~6~~7~~8}\\
\mbox{~9~~10~11~12~13~14~15}\\
\mbox{~16~17~18~19~20~21~22}\\
\mbox{~23~24~25~26~27~28~29}\\
\mbox{~30~31}\\
\mbox{~==~==~==~==~==~==~==}\\
\end{flushleft}\end{ttfamily}

Will be shown as:

\setlongtables
\begin{center}
\begin{longtable}[c]{|p{0.12\\textwidth}|p{0.12\\textwidth}|p{0.12\\textwidth}|p{0.12\\textwidth}|p{0.12\\textwidth}|p{0.12\\textwidth}|p{0.12\\textwidth}|}\hline

su
&
mo
&
tu
&
we
&
th
&
fr
&
sa

 \\ \hline
\endhead
&
&
&
&
&

&
1
\\ \hline
2
&
3
&
4
&
5
&
6
&
7
&
8
\\ \hline
9
&
10
&
11
&
12
&
13
&
14
&
15
\\ \hline
16
&
17
&
18
&
19
&
20
&
21
&
22
\\ \hline
23
&
24
&
25
&
26
&
27
&
28
&
29
\\ \hline
30
&
31
&
&
&
&
&
\\ \hline
\end{longtable}
\end{center}

\hypertarget{lcomment}{}

\hypertarget{lcomments}{}
\subsection{Comments}

To comment fragment of \TeX{}t it should be prepended with two dots:

\begin{ttfamily}\begin{flushleft}
\mbox{~..~This~\TeX{}t~will~be~commented.}\\
\mbox{}\\
\mbox{~~~~This~\TeX{}t~also~will~be~commented.}\\
\mbox{}\\
\mbox{~But~this~not.}\\
\end{flushleft}\end{ttfamily}

To make commenting easier dots can be in previous line:

\begin{ttfamily}\begin{flushleft}
\mbox{~..}\\
\mbox{~~~~~This~line~will~be~commented.}\\
\mbox{}\\
\mbox{~And~this~not.}\\
\end{flushleft}\end{ttfamily}

However, when line with two dots (and only two dots, eventually spaces) will
be followed by blank line even indented lines won't be commented out.

Comments may be useful to place in output code useful things like Vim
modelines:

\begin{ttfamily}\begin{flushleft}
\mbox{~..~vim:set~tw=72:}\\
\end{flushleft}\end{ttfamily}

or folding markers:

\begin{ttfamily}\begin{flushleft}
\mbox{~..~\{\{\{}\\
\mbox{}\\
\mbox{~..~\}\}\}}\\
\end{flushleft}\end{ttfamily}

These lines will be in exported format, just not visible.

\hypertarget{lfootnotes}{}
\section{Footnotes}

You can include in \TeX{}t special links to fragments which don't match
into current paragraph, and place those fragments wherever you want in
document. Vim reStructured \TeX{}t supports three types of footnotes: \textbf{numbered}, \textbf{labeled}
and \textbf{auto-numbered}.

It is possible to use many structure elements in footnotes. They have to be
indented up to ``[`` opening footnote declaration.

\TeX{}t of footnotes cannot be placed inside of tables.

\hypertarget{lnumbered-footnotes}{}
\subsection{Numbered footnotes}

The simplest one. Number is manually assigned to footnote. Example:

\begin{ttfamily}\begin{flushleft}
\mbox{~This~doesn't~belong~here[1]\_.}\\
\mbox{}\\
\mbox{~..~[1]~I~will~describe~it~here.}\\
\end{flushleft}\end{ttfamily}

Results in:

This doesn't belong here\footnote{I will describe it here.
}.

There are numbered footnotes, try to keep them in order to not
disorientate readers. Vim reStructured \TeX{}t will not warn about duplicate footnotes.

\hypertarget{lauto-numbered-footnotes}{}
\subsection{Auto-numbered footnotes}

In this type footnotes are declared by author only with \texttt{[\#]\_}. \TeX{}t of
footnote will look like:

\begin{ttfamily}\begin{flushleft}
\mbox{~..~[\#]~Footnote~\TeX{}t.}\\
\end{flushleft}\end{ttfamily}

Order of \texttt{\#} signs is very important. First \texttt{[\#]\_} will be
connected with first \texttt{.. [\#]}, second \texttt{[\#]\_} with second \texttt{.. [\#]}
and so on.

\hypertarget{llabeled-footnotes}{}
\subsection{Labeled footnotes}

Marking footnotes with \texttt{[\#]\_} is fast but user can easily lost orientation.
Solution can be use of labeled footnotes. Example:

\begin{ttfamily}\begin{flushleft}
\mbox{~This~is~labeled[\#lfoot]\_~footnote.}\\
\mbox{}\\
\mbox{~..~[\#lfoot]~Labeled~footnote~looks~similar~to~auto~numbered~but~\#~is}\\
\mbox{~~~~followed~by~short~alphanumeric~description.}\\
\end{flushleft}\end{ttfamily}

\hypertarget{lmixing-of-footnotes}{}
\subsection{Mixing of footnotes}

User can mix types of footnotes but results can be unexpected (and for sure
they will be different from \href{http://docutils.sf.net}{reST}).

First numbers will be assigned to numbered footnotes, labeled footnotes will
follow with first number bigger then maximum number of numbered footnotes.
Numbers to labels will be assigned in order of use of labels in \TeX{}t. The last
ones will be auto-numbered footnotes.

\begin{center}
\fbox{\begin{minipage}{0.8\textwidth}
\textbf{\sffamily\large Note}
\vspace{2mm}

Numbering of footnotes in \LaTeX{} will strictly follow order of
footnotes. With mixed types is high probability numbers of footnotes in two
types of export will be different.
\end{minipage}}
\end{center}
\hypertarget{lcitations}{}
\section{Citations}

Special case of \href{\#lfootnotes}{footnotes} are citations. In form:

\begin{ttfamily}\begin{flushleft}
\mbox{~This~is~citation~[Smith1995]\_}\\
\mbox{}\\
\mbox{~..~[Smith1995]~John~Smith,~*Something~about~nothing*,~Kein~Press,~1995.}\\
\end{flushleft}\end{ttfamily}

Will create footnote-like paragraph and link to this paragraph.

\hypertarget{ldirectives}{}
\section{Directives}

\subtitle{Or how to place special elements in \TeX{}t
}

You can achieve special treating of some paragraphs by using
directives. They have always a form of:

\begin{ttfamily}\begin{flushleft}
\mbox{~..~directive::}\\
\end{flushleft}\end{ttfamily}

Some directives can contain many various elements of \TeX{}t like lists,
preformatted \TeX{}t, even other directives.

Unknown directives will be displayed in red frame and in monospace font.

\hypertarget{limages}{}
\subsection{Images}

To include an image in your document, you use the \texttt{image} directive.
For example:

\begin{ttfamily}\begin{flushleft}
\mbox{~..~image::~test.png}\\
\end{flushleft}\end{ttfamily}

Spaces in filename should be avoided (or encoded as \%20, but can work properly
on standard settings.

\begin{itemize}
\item
\textbf{Argument}: path to image

\item
\textbf{Options}:

 \begin{deflist}{iii}

\item[ \texttt{:width:}]

Sets width of image in output document. Example:

\begin{ttfamily}\begin{flushleft}
\mbox{~:width:~200}\\
\end{flushleft}\end{ttfamily}

\item[ \texttt{:height:}]

Sets height of image in output document. Example:

\begin{ttfamily}\begin{flushleft}
\mbox{~:height:~100}\\
\end{flushleft}\end{ttfamily}

\item[ \texttt{:identify:}]

Calls \texttt{identify} program from \href{http://www.imagemagick.org}{ImageMagick} suite to identify
dimensions of image. Possible use of argument -- number will be value
how to scale image in percents. When containing non digit chars,
ignored. Example:

\begin{ttfamily}\begin{flushleft}
\mbox{~:identify:~50}\\
\end{flushleft}\end{ttfamily}

 \begin{center}
\fbox{\begin{minipage}{0.8\textwidth}
 \textbf{\sffamily\large Note}
\vspace{2mm}

Not available in reST.
 \end{minipage}}
\end{center}

\item[ \texttt{:scale:}]

Scale values from \texttt{width}, \texttt{height} and/or \texttt{identify}. Ignored
when values not supplied or argument contain non digit chars. Example:

\begin{ttfamily}\begin{flushleft}
\mbox{~:scale:~50}\\
\end{flushleft}\end{ttfamily}

\item[ \texttt{:alt:}]

Alternative \TeX{}t to show in WWW browsers when image not loaded. HTML
export only. Example:

\begin{ttfamily}\begin{flushleft}
\mbox{~:alt:~Alternative~\TeX{}t}\\
\end{flushleft}\end{ttfamily}

\item[ \texttt{:title:}]

Title of image to show in WWW browsers and as caption of image in
\LaTeX{}/PDF output. Example:

\begin{ttfamily}\begin{flushleft}
\mbox{~:title:~Title~of~image}\\
\end{flushleft}\end{ttfamily}

\item[ \texttt{:target:}]

Makes image a link. Argument is a path to location. Special argument
\texttt{self} points to image itself. Examples:

\begin{ttfamily}\begin{flushleft}
\mbox{~:target:~URL}\\
\mbox{~:target:~start\_}\\
\mbox{~:target:~self}\\
\end{flushleft}\end{ttfamily}

\item[ \texttt{:align:}]

Moves image to the side of document making \TeX{}t flowing around it.
Allowed arguments are \texttt{right} and \texttt{left}. HTML export only.

\item[ \texttt{:class:}]

Apply special class to image. HTML export only.
\end{deflist}

\item
\textbf{Content}: NONE
\end{itemize}
\hypertarget{limage-examples}{}
\subsubsection{Image examples}

You can supply additional information about image with options:

\begin{ttfamily}\begin{flushleft}
\mbox{~..~image::~test.png}\\
\mbox{~~~~:width:~200}\\
\mbox{~~~~:height:~100}\\
\mbox{~~~~:alt:~Alternative~\TeX{}t}\\
\mbox{~~~~:title:~Title~of~image}\\
\end{flushleft}\end{ttfamily}

Results in:

\begin{figure}[ht]\centering\includegraphics[height=100pt, width=200pt]{test.png}\caption{Title of image}\end{figure}
Getting info about image dimensions is boring. You can use special
option \texttt{:identify:} which uses program from \href{http://www.imagemagick.org}{ImageMagick} suite
of programs (available on most OS where Vim is available):

\begin{ttfamily}\begin{flushleft}
\mbox{~..~image::~test.png}\\
\mbox{~~~~:identify:}\\
\mbox{~~~~:alt:~Alternative~\TeX{}t}\\
\mbox{~~~~:title:~Title~of~image}\\
\end{flushleft}\end{ttfamily}

\texttt{identify:} can handle argument which will serve as scale factor. 100
is scale 1:1, scale will decrease size of image \textbf{only in document}. Real
size of image will not change:

\begin{ttfamily}\begin{flushleft}
\mbox{~..~image::~test.png}\\
\mbox{~~~~:identify:~50}\\
\mbox{~~~~:alt:~Alternative~\TeX{}t}\\
\mbox{~~~~:title:~Title~of~image}\\
\end{flushleft}\end{ttfamily}

Similar effect can be achieved with option \texttt{:scale:}. Note that
\texttt{:identify:} argument and \texttt{:scale:} will accumulate. If you declare 50 in
both image will have only 25\% of linear size.

It is possible to make image a link with option \texttt{:target:}:

\begin{ttfamily}\begin{flushleft}
\mbox{~..~image::~test.png}\\
\mbox{~~~~:target:~http://www.vst.info/test.png}\\
\end{flushleft}\end{ttfamily}

It will make image a link to other image. When you are scaling image
view it is a good idea to make it clickable and point to full scale
version -- possible with special argument \texttt{self} (note limited
usability in \LaTeX{} export):

\begin{ttfamily}\begin{flushleft}
\mbox{~..~image::~test.png}\\
\mbox{~~~~:identify:~50}\\
\mbox{~~~~:target:~self}\\
\mbox{~~~~:alt:~Alternative~\TeX{}t}\\
\mbox{~~~~:title:~Title~of~image}\\
\end{flushleft}\end{ttfamily}

Results in:

\begin{figure}[ht]\centering\href{test.png}{\includegraphics{test.png}}\caption{Title of image}\end{figure}
\hypertarget{lfigures}{}
\subsection{Figures}

Figure is special construction which creates image with following \TeX{}t
elements will be placed in separate frame with possible \TeX{}t flowing around
(HTML only):

\begin{ttfamily}\begin{flushleft}
\mbox{~..~figure::~test.png}\\
\mbox{~~~~:identify:}\\
\mbox{}\\
\mbox{~~~~This~is~description~of~this~figure.}\\
\mbox{}\\
\mbox{~~~~1.~Can~use~}\\
\mbox{~~~~2.~Different~elements}\\
\end{flushleft}\end{ttfamily}

Spaces in filename should be avoided (or encoded as \%20, but can work properly
on standard settings.

\begin{itemize}
\item
\textbf{Argument}: path to image

\item
\textbf{Options}:

 \begin{deflist}{iii}

\item[ \texttt{:width:}]

Sets width of image in output document. Example:

\begin{ttfamily}\begin{flushleft}
\mbox{~:width:~200}\\
\end{flushleft}\end{ttfamily}

\item[ \texttt{:height:}]

Sets height of image in output document. Example:

\begin{ttfamily}\begin{flushleft}
\mbox{~:height:~100}\\
\end{flushleft}\end{ttfamily}

\item[ \texttt{:identify:}]

Calls \texttt{identify} program from \href{http://www.imagemagick.org}{ImageMagick} suite to identify
dimensions of image. Possible use of argument -- number will be value
hot to scale image. When containing non digit chars, ignored.
Example:

\begin{ttfamily}\begin{flushleft}
\mbox{~:identify:~50}\\
\end{flushleft}\end{ttfamily}

 \begin{center}
\fbox{\begin{minipage}{0.8\textwidth}
 \textbf{\sffamily\large Note}
\vspace{2mm}

Not available in reST.
 \end{minipage}}
\end{center}

\item[ \texttt{:scale:}]

Scale values from \texttt{width}, \texttt{height} and/or \texttt{identify}. Ignored
when values not supplied or argument contain non digit chars. Example:

\begin{ttfamily}\begin{flushleft}
\mbox{~:scale:~50}\\
\end{flushleft}\end{ttfamily}

\item[ \texttt{:alt:}]

Alternative \TeX{}t to show in WWW browsers when image not loaded. HTML
export only. Example:

\begin{ttfamily}\begin{flushleft}
\mbox{~:alt:~Alternative~\TeX{}t}\\
\end{flushleft}\end{ttfamily}

\item[ \texttt{:title:}]

Title of image to show in WWW browsers and as caption of image in
\LaTeX{}/PDF output. Example:

\begin{ttfamily}\begin{flushleft}
\mbox{~:title:~Title~of~image}\\
\end{flushleft}\end{ttfamily}

\item[ \texttt{:target:}]

Makes image a link. Argument is a path to location. Special argument
\texttt{self} points to image itself. Examples:

\begin{ttfamily}\begin{flushleft}
\mbox{~:target:~URL}\\
\mbox{~:target:~start\_}\\
\mbox{~:target:~self}\\
\end{flushleft}\end{ttfamily}

\item[ \texttt{:align:}]

Moves image to the side of document making \TeX{}t flowing around it.
Allowed arguments are \texttt{right} and \texttt{left}. HTML export only.

 \begin{center}
\fbox{\begin{minipage}{0.8\textwidth}
 \textbf{\sffamily\large Note}
\vspace{2mm}

When in options without \texttt{:figalign:} will be interpreted
as align of figure, not image.
 \end{minipage}}
\end{center}

\item[ \texttt{:class:}]

Apply special class to image. HTML export only.

\item[ \texttt{:figwidth:}]

Width of figure. By default 400px on HTML export and 0.6 of \TeX{}twidth
in \LaTeX{}. HTML export only.

\item[ \texttt{:figalign:}]

Side where figure will be placed and \TeX{}t will flow around it. HTML
export only.

\item[ \texttt{:figclass:}]

Apply special class to figure. HTML export only.
\end{deflist}

\item
\textbf{Content}: Interpreted as body elements.
\end{itemize}
\hypertarget{ltopic}{}
\subsection{Topic}

A topic is like a block quote with a title, or a self-contained section with
no subsections. Use the "topic" directive to indicate a self-contained idea
that is separate from the flow of the document. Topics may occur anywhere
a section or transition may occur.

The directive's sole argument is interpreted as the topic title; the next line
must be blank. All subsequent lines make up the topic body, interpreted as
body elements. Example:

\begin{ttfamily}\begin{flushleft}
\mbox{~..~topic::~Header~of~topic}\\
\mbox{}\\
\mbox{~~~~~~These~lines~are~topic~content~interpreted}\\
\mbox{~~~~~~as~body~elements.}\\
\end{flushleft}\end{ttfamily}

\begin{itemize}
\item
\textbf{Argument}: header of topic

\item
\textbf{Options}:

 \begin{deflist}{iii}

\item[ \texttt{:class:}]

Name of class applied to the topic. Only in HTML export.

 One class is predefined -- sidebar:

\begin{ttfamily}\begin{flushleft}
\mbox{~..~topic::~Notes~on~margin}\\
\mbox{~~~~:class:~sidebar}\\
\end{flushleft}\end{ttfamily}

 Elements from that topic will be put in float on right margin.
\end{deflist}

\item
\textbf{Content}: Interpreted as body elements
\end{itemize}
\hypertarget{lsidebar}{}
\subsection{Sidebar}

A sidebar is like a block quote with a title (also can be subtitle). Use the
"sidebar" directive to indicate a self-contained idea that is separate from
the flow of the document.

The directive's sole argument is interpreted as the sidebar title; the next
line must be blank. All subsequent lines make up the sidebar body, interpreted
as body elements. Example:

\begin{ttfamily}\begin{flushleft}
\mbox{~..~sidebar::~Header~of~subtitle}\\
\mbox{~~~~:subtitle:~Why~this~is~important}\\
\mbox{}\\
\mbox{~~~~~~These~lines~are~sidebar~content~interpreted}\\
\mbox{~~~~~~as~body~elements.}\\
\end{flushleft}\end{ttfamily}

\begin{itemize}
\item
\textbf{Argument}: header of sidebar

\item
\textbf{Options}:

 \begin{deflist}{iii}

\item[ \texttt{:class:}]

Name of class applied to the topic. Only in HTML export.

\item[ \texttt{:subtitle:}]

Subtitle of topic.
\end{deflist}

\item
\textbf{Content}: Interpreted as body elements
\end{itemize}
\hypertarget{ltable-of-contents}{}
\subsection{Table of contents}

For longer \TeX{}t it is good idea to put in document table of contents.
In Vim reStructured \TeX{}t you can place table of contents at desired position with
directive:

\begin{ttfamily}\begin{flushleft}
\mbox{~..~contents::}\\
\end{flushleft}\end{ttfamily}

In exported document it will be replaced by unordered list with elements
indented to present structure of document.

When contents directive is used sections headers will become links to
corresponding entries in table of contents.

\begin{itemize}
\item
\textbf{Argument}: This directive automatically places also title. Default is
'Contents'. If you want give title in your language add this as argument to
directive (example for Polish):

\begin{ttfamily}\begin{flushleft}
\mbox{~..~contents::~Spis~tresci}\\
\end{flushleft}\end{ttfamily}

 I have omitted Polish diacritics to avoid encoding problems.

\item
\textbf{Options}:

 \begin{deflist}{iii}

\item[ \texttt{:depth:}]

Directive \texttt{contents} accepts option \texttt{:depth:} which
argument is level of headers shown in table of contents:

\begin{ttfamily}\begin{flushleft}
\mbox{~..~contents:}\\
\mbox{~~~~:depth:~3}\\
\end{flushleft}\end{ttfamily}

 Will show in table of contents only headers down to 3\sups{rd}
 level.

\item[ \texttt{:class:}]

Class of table of contents. In HTML CSS it is presented by two
elements: \texttt{span} (TOC header) and \texttt{ul} (TOC contents).
\end{deflist}

\item
\textbf{Content}: NONE
\end{itemize}

In HTML export table of contents will be numbered thanks to \texttt{content}
property. Alas, only small number of WWW browsers are supporting this
feature (\href{http://www.kde.org}{Konqueror}, \href{http://www.opera.com}{Opera}, Firefox1.5).

\hypertarget{lreplacement}{}
\subsection{Replacement}

This is an exception from general format of directive:

\begin{ttfamily}\begin{flushleft}
\mbox{~..~|from|~replace::~into}\\
\end{flushleft}\end{ttfamily}

It consists of four parts: leading commas; source part enclosed in
bars; name of directive -- \texttt{replace::}; and the rest of line which will
replace source in file (without leading space). Beware: \texttt{from} and \texttt{into}
elements are processed through \texttt{substitute()} Vim function and have to be
proper Vim regexps. Three characters will be escaped automatically: \texttt{$\backslash$ \& \~{}}.

\hypertarget{linline-images}{}
\subsubsection{Inline images}

Special type of replacement, designed for placement of images inline and
in \href{\#ltables}{tables}.

In \TeX{}t you can use it as normal \href{\#lreplacement}{replacement} but declaration is
different:

\begin{ttfamily}\begin{flushleft}
\mbox{~..~|from|~image::image.jpg}\\
\mbox{~~~~:width:~20~}\\
\mbox{~~~~:height:~20~}\\
\mbox{~~~~:target:~URL}\\
\end{flushleft}\end{ttfamily}

All options of \href{\#limages}{Images} directive are supported.

Result looks like that: \href{test.png}{\includegraphics[height=20pt, width=20pt]{test.png}}.

\hypertarget{lunicode}{}
\subsubsection{Unicode}

Not all characters can be shown in localized encodings. For the rest it is
possible to use Unicode (U+ followed by hexadecimal number):

\begin{ttfamily}\begin{flushleft}
\mbox{~..~|from|~unicode::~U+2211~..~Sigma~sign}\\
\end{flushleft}\end{ttfamily}

This \TeX{}t will replace marker into unicode sigma sign. Encoding of output
document, regardless of current Vim encoding will be set to utf-8. \TeX{}t after
two dots will be ignored.

User can put longer \TeX{}t in replacement:

\begin{ttfamily}\begin{flushleft}
\mbox{~..~|200E|~unicode::~200~U+20AC~..~200~euro}\\
\end{flushleft}\end{ttfamily}

Spaces will be removed from output.

Unicode directive accepts one of three options:

\begin{deflist}{iii}

\item[\texttt{:ltrim:}]

Will remove space from the left side of sign.

\item[\texttt{:trim:}]

Will remove space from left and right side of sign.

\item[\texttt{:rtrim:}]

Will remove space from right side of sign.
\end{deflist}
\begin{center}
\fbox{\begin{minipage}{0.8\textwidth}
\textbf{\sffamily\large Warning}
\vspace{2mm}

 \begin{itemize}
\item
Encoding of Vim will not be changed and characters encoded in
utf-8 may be unreadable on terminal with non utf-8 encoding.

\item
Unicode replacements doesn't work for standard \LaTeX{}
configuration. Direct PDF export will not work and \LaTeX{} may need special
configuration.
\end{itemize}
\end{minipage}}
\end{center}
\hypertarget{ldate}{}
\subsubsection{Date}

Inserting of date is tedious. To make it simpler use date replacement
directive:

\begin{ttfamily}\begin{flushleft}
\mbox{~..~|date|~date::}\\
\mbox{}\\
\mbox{~..~|time|~date::~\%H:\%M}\\
\end{flushleft}\end{ttfamily}

Without arguments placeholder will be replaced with date in ISO format
YYYY-MM-DD. Arguments are following \texttt{strftime()} syntax. This function isn't
available on all systems, in such case placeholder will be replaced with
seconds from epoch :)

\hypertarget{lincluding-commands}{}
\subsection{Including commands}

With bigger sets of documents some parts can be repeated. In such case it is
good idea to put them in external file and only include it in proper place.
Also it gives profits when making changes. User have to only make correction
in one file, not in whole collection. Vim reStructured \TeX{}t provides three directives. All are
making the same thing -- include \TeX{}t file -- written in VST before any other
activity.

\begin{enumerate}[label=\arabic*.]
\item
This command will include file in given place:

\begin{ttfamily}\begin{flushleft}
\mbox{~..~header::~\{filename\}}\\
\end{flushleft}\end{ttfamily}

 Regardless of placement of this directive file will be put in the
 beginning of exported file.

\item
This command will include file in given place:

\begin{ttfamily}\begin{flushleft}
\mbox{~..~include::~\{filename\}}\\
\end{flushleft}\end{ttfamily}

\item
This command will include file in given place:

\begin{ttfamily}\begin{flushleft}
\mbox{~..~footer::~\{filename\}}\\
\end{flushleft}\end{ttfamily}

 Regardless of placement of this directive file will be put at the end of
 exported file.
\end{enumerate}
\begin{itemize}
\item
\textbf{Argument}: String.

 \begin{itemize}
\item
If it is readable file its contents will be included and parsed as Vim reStructured \TeX{}t.

\item
Predefined name: \texttt{vstfooter}. Contains date of generation and link to
source file. \texttt{vstfooter} will be used also in case when argument will be
empty.

\item
When filename is enclosed in $<$$>$ Vim reStructured \TeX{}t will search in directory defined by
\texttt{g:vst\_included} variable. By default it is defined as: first element of
'runtimepath' option plus \texttt{autoload/vst/include/} directory. For example
on Linux it will be:

\begin{ttfamily}\begin{flushleft}
\mbox{~\$HOME/.vim/autoload/vst/include/}\\
\end{flushleft}\end{ttfamily}

 This can be used for standard substitutions like provided by \href{http://docutils.sf.net/docs/ref/rst/substitutions.html\#character-entity-sets}{reST
 substitutes}, personal set of substitutions or other type of data.

\item
For header and footer in other cases it will be treated as short message
to include at the top or bottom of the document and separated with
horizontal line from the rest of document.

\item
When including files indentation of directive will be taken into account.
It may be convenient to produce program listings in connection with \href{\#ldouble-colon}{double
colon} or \href{\#l2html}{2html} directive.
\end{itemize}

\item
\textbf{Options}: NONE

\item
\textbf{Content}: NONE
\end{itemize}
\begin{center}
\fbox{\begin{minipage}{0.8\textwidth}
\textbf{\sffamily\large Note}
\vspace{2mm}

To avoid endless loops level of recursiveness is equal to Vim's
option 'maxfuncdepth'/2 (default: 50).
\end{minipage}}
\end{center}

\hypertarget{ltip}{}
\subsection{Tip}

This directive can contain many various elements:

\begin{ttfamily}\begin{flushleft}
\mbox{~..~tip::~First~paragraph.}\\
\mbox{}\\
\mbox{~~~~~1.~List~element}\\
\mbox{~~~~~2.~List~element}\\
\mbox{}\\
\mbox{~~~~~Second~paragraph}\\
\end{flushleft}\end{ttfamily}

\begin{itemize}
\item
\textbf{Argument}: NONE

\item
\textbf{Options}: NONE

\item
\textbf{Content}: Interpreted as body elements
\end{itemize}

Above example results in:

\begin{center}
\fbox{\begin{minipage}{0.8\textwidth}
\textbf{\sffamily\large Tip}
\vspace{2mm}

First paragraph.

 \begin{enumerate}[label=\arabic*.]
\item
List element

\item
List element
 \end{enumerate}

 Second paragraph
\end{minipage}}
\end{center}

All elements must have bigger indentation than directive. These
elements will be placed in green frame and with title 'Tip'.

\hypertarget{lnote}{}
\subsection{Note}

This directive can contain many various elements:

\begin{ttfamily}\begin{flushleft}
\mbox{~..~Note::~First~paragraph.}\\
\mbox{}\\
\mbox{~~~~~~~~Remember~these~noble~words.}\\
\mbox{}\\
\mbox{~~~~Second~paragraph}\\
\end{flushleft}\end{ttfamily}

\begin{itemize}
\item
\textbf{Argument}: NONE

\item
\textbf{Options}: NONE

\item
\textbf{Content}: Interpreted as body elements
\end{itemize}

Above example results in:

 \begin{center}
\fbox{\begin{minipage}{0.8\textwidth}
 \textbf{\sffamily\large Note}
\vspace{2mm}

First paragraph.

 \begin{quotation}
 Remember these noble words.
 \end{quotation}

 Second paragraph
\end{minipage}}
\end{center}

All elements must have bigger indentation than directive. These
elements will be placed in black frame and with title 'Note'.

\hypertarget{lwarning}{}
\subsection{Warning}

This directive can contain many various elements:

\begin{ttfamily}\begin{flushleft}
\mbox{~..~Warning::~First~paragraph.}\\
\mbox{}\\
\mbox{~~~~~-~List~element}\\
\mbox{~~~~~-~List~element}\\
\mbox{}\\
\mbox{~~~~~Second~paragraph}\\
\end{flushleft}\end{ttfamily}

\begin{itemize}
\item
\textbf{Argument}: NONE

\item
\textbf{Options}: NONE

\item
\textbf{Content}: Interpreted as body elements
\end{itemize}

Above example results in:

\begin{center}
\fbox{\begin{minipage}{0.8\textwidth}
\textbf{\sffamily\large Warning}
\vspace{2mm}

First paragraph.

 \begin{itemize}
\item
List element

\item
List element
 \end{itemize}

 Second paragraph
\end{minipage}}
\end{center}

All elements must have bigger indentation than directive. These
elements will be placed in red frame and with title 'Warning'.

\hypertarget{ladmonition}{}
\subsection{Admonition}

General form of multi element message directive (MEMD). Have form:

\begin{ttfamily}\begin{flushleft}
\mbox{~..~admonition::~<title>}\\
\mbox{~~~~:class:~<class>}\\
\mbox{}\\
\mbox{~~~~\TeX{}t~of~admonition~(many~elements).}\\
\end{flushleft}\end{ttfamily}

\begin{itemize}
\item
\textbf{Argument}: title of admonition

\item
\textbf{Options}

 \begin{deflist}{iii}

\item[ \texttt{:class:}]

describes look of admonition. By default \texttt{note}, available three
classes: \texttt{note}, \texttt{tip}, \texttt{warning}.
\end{deflist}

\item
\textbf{Content}: Interpreted as body elements. MUST be separated from header by
blank line.
\end{itemize}

Directive can be used as i18n version of any other MEMD.

\hypertarget{lpull-quote}{}
\subsection{Pull-quote}

Semantically similar to blockquote but in addition to being indented \TeX{}t will
be displayed with bigger font to attract attention:

\begin{ttfamily}\begin{flushleft}
\mbox{~..~pull-quote::~}\\
\mbox{}\\
\mbox{~~~~This~is~time~of~our~lives!}\\
\end{flushleft}\end{ttfamily}

Results in:

\begin{pullquote}
 This is time of our lives!
\end{pullquote}
\begin{itemize}
\item
\textbf{Argument}: NONE

\item
\textbf{Options}: NONE

\item
\textbf{Content}: Interpreted as block quote.
\end{itemize}
\hypertarget{lcontainer}{}
\subsection{Container}

General block directive:

\begin{ttfamily}\begin{flushleft}
\mbox{~..~container::~classname}\\
\end{flushleft}\end{ttfamily}

\begin{itemize}
\item
\textbf{Argument}: class name, may use any character, string will be parsed to
eliminate them

\item
\textbf{Options}: NONE

\item
\textbf{Content}: Interpreted as body elements
\end{itemize}

HTML output:

\begin{ttfamily}\begin{flushleft}
\mbox{~<div~class="classname">~content~of~container~</div>}\\
\end{flushleft}\end{ttfamily}

\LaTeX{} output:

\begin{ttfamily}\begin{flushleft}
\mbox{~$\backslash$vstclassname\{~content~of~container~\}}\\
\end{flushleft}\end{ttfamily}

Note \texttt{vst} prefix to avoid possible conflicts with built-in (La)\TeX{}
commands. In preamble will be inserted simple template to allow compilation of
document without stopping on unknown commands:

\begin{ttfamily}\begin{flushleft}
\mbox{~$\backslash$newcommand\{$\backslash$vstclassname\}[1]\{\#1\}}\\
\end{flushleft}\end{ttfamily}

It will be inserted before declaration of extension of preamble by \href{\#lvtp}{external
file}. If user wants to declare there these new commands he should
use \texttt{renewcommand} \LaTeX{} command.

\hypertarget{lcompound}{}
\subsection{Compound}

Similar to \href{\#lcontainer}{container} but designed to distinguish \TeX{}t rather semantically, not
visually:

 \vstvstcompound{}
\begin{itemize}
\item
\textbf{Argument}: NONE

\item
\textbf{Options}

 \begin{deflist}{iii}

\item[ \texttt{:class:}]

describes look of admonition
\end{deflist}

\item
\textbf{Content}: Interpreted as body elements
\end{itemize}
\hypertarget{lclass}{}
\subsection{Class}

This directive allows to apply arbitrary class name to most block elements
(HTML only):

\begin{ttfamily}\begin{flushleft}
\mbox{~..~class::~name}\\
\end{flushleft}\end{ttfamily}

There is an exception: Class directive will be ignored for directives which
has \texttt{:class:} option.

If next element has the same level of indentation class will be applied only
to that alement. But if next elements has bigger indentation class will be
applied to all of them:

\begin{ttfamily}\begin{flushleft}
\mbox{~~..~class::~name}\\
\mbox{}\\
\mbox{~~~~~First~paragraph}\\
\mbox{}\\
\mbox{~~~~~Second~paragraph}\\
\mbox{}\\
\mbox{~Third~paragraph}\\
\end{flushleft}\end{ttfamily}

\texttt{name} class will be applied to first and second paragraph but to to third.

\begin{center}
\fbox{\begin{minipage}{0.8\textwidth}
\textbf{\sffamily\large Note}
\vspace{2mm}

\texttt{name} class style will be added to default style, not replace it.
\end{minipage}}
\end{center}
\begin{itemize}
\item
\textbf{Argument}: class name, may use any character, string will be parsed to
eliminate them

\item
\textbf{Options}: NONE

\item
\textbf{Content}: Interpreted as body elements
\end{itemize}
\hypertarget{ldefault-role}{}
\subsection{Default role}

This role sets default \TeX{}t role used for interpreted \TeX{} without declared
role. For example, after setting:

\begin{ttfamily}\begin{flushleft}
\mbox{~..~default-role::~newstyle}\\
\end{flushleft}\end{ttfamily}

\TeX{}t enclosed in backticks will be presented with help of \texttt{newstyle} class.
Default role for Vim reStructured \TeX{}t is \href{\#ltitle-reference}{title reference}.

\begin{itemize}
\item
\textbf{Argument}: class name, may use any character, string will be parsed to
eliminate them

\item
\textbf{Options}: NONE

\item
\textbf{Content}: NONE
\end{itemize}
\hypertarget{lmeta}{}
\subsection{Meta}

This directive can be useful when providing meta data which user don't want to
be visible in form of \href{\#lfield-list}{field list}:

\begin{ttfamily}\begin{flushleft}
\mbox{~..~meta::}\\
\mbox{~~~~:author:~Mikolaj~Machowski}\\
\mbox{~~~~:subject:~Vim~reStructured~\TeX{}t}\\
\mbox{~~~~:http-equiv=Content-Type:~\TeX{}t/html;~charset=iso-8859-2}\\
\mbox{~~~~:description~lang=en:~Special~\TeX{}t~formatting}\\
\end{flushleft}\end{ttfamily}

In HTML export this list will be included into meta tags of header. In \LaTeX{}
four keywords will be recognized and put into PDF info: author, title,
subject, keywords.

\begin{itemize}
\item
\textbf{Argument}: List of keywords with descriptions.

\item
\textbf{Options}: NONE

\item
\textbf{Content}: NONE
\end{itemize}
\hypertarget{ltitle}{}
\subsection{Title}

Title directive may be used to place your own \TeX{}t in title tag of HTML export
or in PDF info fields. This title will not be visible in document itself:

\begin{ttfamily}\begin{flushleft}
\mbox{~..~title::~VST~documentation}\\
\end{flushleft}\end{ttfamily}

\begin{itemize}
\item
\textbf{Argument}: Title of document

\item
\textbf{Options}: NONE

\item
\textbf{Content}: NONE
\end{itemize}
\hypertarget{lrubric}{}
\subsection{Rubric}

This directive shows line as header but without semantic meaning of header:

\begin{ttfamily}\begin{flushleft}
\mbox{~..~rubric::~This~may~be~of~interest~for~readers}\\
\end{flushleft}\end{ttfamily}

\begin{itemize}
\item
\textbf{Argument}: class name, may use any character, string will be parsed to
eliminate them

\item
\textbf{Options}: NONE

\item
\textbf{Content}: NONE
\end{itemize}
\hypertarget{lraw}{}
\subsection{Raw}
\hypertarget{lraw-LaTeX}{}
\subsubsection{Raw \LaTeX{}}

HTML and pure \TeX{}t are often not enough to present some mathematical
concepts. You can use then raw \LaTeX{} directive:

\begin{ttfamily}\begin{flushleft}
\mbox{~..~raw::~\LaTeX{}}\\
\mbox{}\\
\mbox{~..~raw::~\TeX{}}\\
\end{flushleft}\end{ttfamily}

\begin{itemize}
\item
\textbf{Argument}: name of export: html, \LaTeX{} (case sensitive)

\item
\textbf{Options}:

 \begin{deflist}{iii}

\item[ \texttt{:file:}]

External file which will be read into file. Example:

\begin{ttfamily}\begin{flushleft}
\mbox{~:file:~path/to/file.\TeX{}}\\
\end{flushleft}\end{ttfamily}

 Contrary to treating of content there is no fall back for other
 format.
\end{deflist}

\item
\textbf{Content}: one indented paragraph of \LaTeX{} source
\end{itemize}

Content of this directive will not be visible in HTML export and in \LaTeX{}
literally. This directive must be a one paragraph of Vim reStructured \TeX{}t -- without blank
lines. But you can make it multi paragraph \LaTeX{} content with \texttt{$\backslash$par}
command:

\begin{ttfamily}\begin{flushleft}
\mbox{~..~raw::~\LaTeX{}}\\
\mbox{}\\
\mbox{~~~~~~~~~~~~~This~is~first~par~with~special~\$$\backslash$pi\$~content.}\\
\mbox{~~~~~$\backslash$par}\\
\mbox{~~~~~~~~~~~~~This~is~second~par~with~special~\$$\backslash$alpha\$~content.}\\
\mbox{~~~~~$\backslash$par}\\
\mbox{~~~~~~~~~~~~~$\backslash$emph\{Indentation\}~doesn't~have~special~meaning~but~it~is}\\
\mbox{~~~~~~~~~~~~~good~for~visual~separation~of~$\backslash$\TeX{}ttt\{paragraphs\}.}\\
\end{flushleft}\end{ttfamily}

Results in:

$<$vim:raw\LaTeX{}$>$
 This is first par with special \$$\backslash$pi\$ content.
 $\backslash$par
 This is second par with special \$$\backslash$alpha\$ content.
 $\backslash$par
 $\backslash$emph\{Indentation\} doesn't have special meaning but it is
 good for visual separation of $\backslash$\TeX{}ttt\{paragraphs\}.
$<$/vim:raw\LaTeX{}$>$

Content of this directive (and above example) will not be visible in HTML
export. This directive must be a one paragraph of Vim reStructured \TeX{}t -- without blank lines.

Raw \LaTeX{} replaces \texttt{\LaTeX{}only} directive which is considered deprecated.

\hypertarget{lraw-html}{}
\subsubsection{Raw html}

\begin{ttfamily}\begin{flushleft}
\mbox{~..~raw::~html}\\
\end{flushleft}\end{ttfamily}

\begin{itemize}
\item
\textbf{Argument}: name of export: html, \LaTeX{} (case sensitive)

\item
\textbf{Options}:

 \begin{deflist}{iii}

\item[ \texttt{:file:}]

External file which will be read into file. Example:

\begin{ttfamily}\begin{flushleft}
\mbox{~:file:~path/to/file.html}\\
\end{flushleft}\end{ttfamily}

 Contrary to treating of content there is no fall back for other
 format.
\end{deflist}

\item
\textbf{Content}: one indented paragraph
\end{itemize}

\begin{ttfamily}\begin{flushleft}
\mbox{~..~raw::~html}\\
\mbox{}\\
\mbox{~~~~~~~<p>CO<sub>2</sub></p>}\\
\end{flushleft}\end{ttfamily}

Results in:

Content of this directive (and above example) will not be visible in \LaTeX{}
export. This directive must be a one paragraph of Vim reStructured \TeX{}t -- without blank lines.

Raw html replaces \texttt{htmlonly} directive which is considered deprecated.

\hypertarget{lboth}{}
\subsubsection{Both}

Directive will accept also two arguments (HTML and \LaTeX{}), it will be passed
without any modifications in two exports:

\begin{ttfamily}\begin{flushleft}
\mbox{~..~raw::~html~\LaTeX{}}\\
\end{flushleft}\end{ttfamily}

Order is not significant.

\hypertarget{l2html}{}
\subsection{2html}
\begin{center}
\fbox{\begin{minipage}{0.8\textwidth}
\textbf{\sffamily\large Note}
\vspace{2mm}

Not available in reST. Colors in HTML only.
\end{minipage}}
\end{center}

Directive designed to make use from \texttt{2html.vim} script for syntax
highlighting of code:

\begin{ttfamily}\begin{flushleft}
\mbox{~..~2html::~[\{filetype\}]}\\
\mbox{}\\
\mbox{~..~2html::~filetype~[\{colorscheme\}]}\\
\end{flushleft}\end{ttfamily}

\texttt{2html} directive is extended declaration of \href{\#lempty-double-colon}{empty double colon}.
Following paragraphs have to be separated by blank line and must have
bigger indentation.

\begin{itemize}
\item
\textbf{Arguments}:

 \begin{deflist}{iii}

\item[ \texttt{filetype} ]

Argument will set proper highlighting for the following fragment of
code (usually those snippets will be too short for automatic
recognition). May be omitted, following paragraphs will be treated
as normal code snippets -- without coloring. Obligatory if you want
to declare \texttt{colorscheme}.

\item[ \texttt{colorscheme}]

Will set color scheme used for syntax highlighting. If omitted
current color scheme will be used. If you started without color
scheme and declared \texttt{colorscheme} argument it will be set. When
declared colorscheme doesn't exist default/current colorscheme will
be used.
\end{deflist}

\item
\textbf{Options}: NONE

\item
\textbf{Content}: Interpreted as preformatted \TeX{}t
\end{itemize}
\begin{center}
\fbox{\begin{minipage}{0.8\textwidth}
\textbf{\sffamily\large Tip}
\vspace{2mm}

Not all colorschemes look good when exported to HTML. It is wise idea
to check result before official presentation. Especially \texttt{default} is
hard to read on white background -- and this is default combination when
doing export from non-GUI version of Vim. It is a good idea to set
\texttt{g:colors\_name} somewhere because it will be used in such case. When you
try to export this document on your system check example below with manxome
colorscheme which is not included in default Vim distribution.
\end{minipage}}
\end{center}
\hypertarget{l2html-examples}{}
\subsubsection{2html examples}

Fragment of \texttt{vst.vim} in blue colorscheme:

\begin{ttfamily}\begin{flushleft}
\mbox{~..~2html::~vim~blue}\\
\mbox{}\\
\mbox{~~~~~if~exists('depth')~\&\&~depth~!=~''}\\
\mbox{~~~~~~~~~let~hdepth~=~strpart(g:ptype[j],~'1')}\\
\mbox{~~~~~~~~~if~hdepth~>~depth}\\
\mbox{~~~~~~~~~~~~~let~j~+=~1}\\
\mbox{~~~~~~~~~~~~~continue}\\
\mbox{~~~~~~~~~endif}\\
\mbox{~~~~~endif}\\
\end{flushleft}\end{ttfamily}

Result:

\begin{ttfamily}\begin{flushleft}
\mbox{~if~exists('depth')~\&\&~depth~!=~''}\\
\mbox{~~~~~let~hdepth~=~strpart(g:ptype[j],~'1')}\\
\mbox{~~~~~if~hdepth~>~depth}\\
\mbox{~~~~~~~~~let~j~+=~1}\\
\mbox{~~~~~~~~~continue}\\
\mbox{~~~~~endif}\\
\mbox{~endif}\\
\end{flushleft}\end{ttfamily}

The same fragment in manxome:

\begin{ttfamily}\begin{flushleft}
\mbox{~if~exists('depth')~\&\&~depth~!=~''}\\
\mbox{~~~~~let~hdepth~=~strpart(g:ptype[j],~'1')}\\
\mbox{~~~~~if~hdepth~>~depth}\\
\mbox{~~~~~~~~~let~j~+=~1}\\
\mbox{~~~~~~~~~continue}\\
\mbox{~~~~~endif}\\
\mbox{~endif}\\
\end{flushleft}\end{ttfamily}

And fragment of Vim itself -- ex\_cmds.c by murphy. Declaration:

\begin{ttfamily}\begin{flushleft}
\mbox{~..~2html::~c~murphy}\\
\end{flushleft}\end{ttfamily}

\begin{ttfamily}\begin{flushleft}
\mbox{~if~(fp\_out~!=~NULL)}\\
\mbox{~\{}\\
\mbox{~(void)mch\_setperm(tempname,}\\
\mbox{~~~~~~~~~~~(long)((st\_old.st\_mode~\&~0777)~|~0600));}\\
\mbox{~/*~this~only~works~for~root:~*/}\\
\mbox{~(void)chown((char~*)tempname,~st\_old.st\_uid,~st\_old.st\_gid);}\\
\mbox{~\}}\\
\end{flushleft}\end{ttfamily}

\hypertarget{lmacros}{}
\section{Macros}

Not everything can be done with VST syntax. Here are coming \textbf{macros}. Macro
is keyword with optional value after colon like:

\begin{ttfamily}\begin{flushleft}
\mbox{~\{macroname:argument\}}\\
\end{flushleft}\end{ttfamily}

They are not codified. To show what can be achieved with them to distribution
is added \texttt{myhtmlvst.vim} file with examples of two macros:

\begin{enumerate}[label=\arabic*.]
\item
\begin{ttfamily}\begin{flushleft}
\mbox{~\{read:file\}}\\
\end{flushleft}\end{ttfamily}

 First macro read file and put it inside of \texttt{pre} tag (it is like Vim
 \texttt{:read} command).

\item
\begin{ttfamily}\begin{flushleft}
\mbox{~\{readpython:file\}}\\
\end{flushleft}\end{ttfamily}

 Second \hypertarget{lmacro}{macro} interpret contents of file and put it also in \texttt{pre} tag
 (like Vim \texttt{:read!} command).

 This command is smart. It can catch name of program between read and : and
 call it later. Thus it can become \{readperl\}, \{readruby\}, \{readbash\}, etc.
 Program name has to match \texttt{$\backslash$w$\backslash$+} regexp.

 \texttt{readbang} has special value -- it will execute shell command.

 \begin{center}
\fbox{\begin{minipage}{0.8\textwidth}
 \textbf{\sffamily\large Warning}
\vspace{2mm}

This macro can have serious security consequences!!!
\end{minipage}}
\end{center}
\end{enumerate}

These two macros are line wise and must be kept in separate lines with
nothing around them.

Macro file is executed with help of \href{\#lvhp}{g:vst\_html\_post variable}.

\hypertarget{lthanks}{}
\section{Thanks}

I'd like to say thanks to:

\begin{itemize}
\item
Bram Moolenaar for great editor

\item
All authors of \texttt{2html.vim}

\item
Authors of \href{http://docutils.sf.net}{reST} documentation

\item
George V. Reilly for extensive testing

\item
Edward G.J. Lee for traditional Chinese support in \LaTeX{}
\end{itemize}
\hypertarget{lfaq}{}
\section{FAQ}
\begin{itemize}
\item
\textbf{Why VST was created?} For long time Vim users were asking for export
to real, human readable HTML \TeX{}t. Default 2html.vim does wonderful job but
only in regard to code. For other documents it is not well suited.

\item
\textbf{Why \href{http://docutils.sf.net}{reST} syntax and not ...?} I was looking through several \TeX{}t format
syntaxes but only reStructured\TeX{}t was looking complete enough for me. Also
major plus for reST was fact that (contrary to all Wikis I saw) it was
designed to look equally good in original \TeX{}t form.

\item
\textbf{Will VST support all reST features?} Rather not. It is too big project
and develop too fast to catch up. All major features at the moment (1 Dec
2005) were implemented, minor features are covered to satisfy most users
IMO -- you can compare differences in \href{http://skawina.eu.org/mikolaj/restdiff.html}{compatibility table}.

 Also there are many pythonisms in reST I am personally not interested. Vim
 Python users will probably install reST and will be happy, other wont be
 interested in them.

\item
\textbf{Will be VST files fully compatible with reST?} It depends on user. There
are features Vim specific (like accepting of Vim commands in option lists),
or features that could not find in reST and seemed necessary for me. Also
some structure elements are practically impossible to implement in the same
way as in reST. I decided it is better to provide different implementation
that nothing.

\item
\textbf{How to make...?} Not everything can be made by pure VST or syntax. That
is what \texttt{g:vst\_xxx\_post} is for. For example, if you really must break
lines in HTML with \texttt{br} tag put placeholder \texttt{\{br\}} in your document,
and write:

\begin{ttfamily}\begin{flushleft}
\mbox{~\%s/\{br\}/<br$\backslash$/>/ge}\\
\end{flushleft}\end{ttfamily}

 in \texttt{myhtmlvst.vim} file and set:

\begin{ttfamily}\begin{flushleft}
\mbox{~let~g:vst\_html\_post~=~"myhtmlvst.vim"}\\
\end{flushleft}\end{ttfamily}

 For other examples check \href{\#lmacros}{macros} section.

\item
\textbf{How to get only body of document?} Yes, this can be useful for writing blogs
or embedding \TeX{}t in other documents. As in previous question the answer is:
\texttt{g:vst\_xxx\_post}. Example for HTML:

\begin{ttfamily}\begin{flushleft}
\mbox{~:1,/<body>/~d}\\
\mbox{~:/<$\backslash$/body/,\$~d}\\
\end{flushleft}\end{ttfamily}

 Also it is possible to get rid of VST specific \texttt{id}s and \texttt{class}es
 with:

\begin{ttfamily}\begin{flushleft}
\mbox{~:\%s/$\backslash$(id$\backslash$|class$\backslash$)=".$\backslash$\{-\}"//ge}\\
\end{flushleft}\end{ttfamily}

\item
\textbf{Why should I choose VST over reST?}

 \begin{enumerate}[label=\arabic*.]
\item
No dependency on external tools. Unpack .zip archive and everything is ready.

\item
reST is hard to install. I tried several times on Linux to set it up
properly -- \$PATH and everything. Failed each time. Don't want even
think how to install it on MS-Windows. Plus installation of Python
itself.

\item
\href{\#l2html}{2html} directive. Whole power of Vim syntax and colorscheme files on
your command!
 \end{enumerate}

 Still, reST users may find some commands provided by VST useful. Check
 \href{\#lauxiliary-commands}{auxiliary commands} section.

\item
\textbf{What are the differences between VST and reST?} Comparison table is long
and is in \href{http://skawina.eu.org/mikolaj/restdiff.html}{separate file}.
\end{itemize}
\hypertarget{lchangelog}{}
\section{ChangeLog}
\begin{deflist}{Keywords:}

\item[FIX:] bug fix

\item[ADD:] new feature

\item[CHG:] change of behaviour

\item[RMV:] remove feature

\item[\LaTeX{}/HTML/etc:] apply to export, without that tag -- global
\end{deflist}

Full ChangeLog is here: \href{http://skawina.eu.org/mikolaj/vst-changelog.html}{vst-changelog}.

Changes from last announcement:

\begin{itemize}
\item
31 Oct 2006 - \textbf{1.3} announcement

 \begin{itemize}
\item
FIX: Vst link broken

\item
FIX: \& in URLs double encoded

\item
FIX: broken chained links with uppercase letters
\end{itemize}

\item
REMEMBER ABOUT UPDATING DOCS
\end{itemize}
\hypertarget{ltodo}{}
\section{TODO}

There is no roadmap for these things, they are just loose thoughts what
can be done in the future.

reST compatibility things are in \href{http://skawina.eu.org/mikolaj/restdiff.html}{compatibility table} with (td) note.

\begin{itemize}
\item
rotation of ornaments in titles

\item
option for XHTML doctype with proper content?

\item
image inside of admonition in \LaTeX{} export (minipage conflicts...)

\item
show worded reference as label

\item
(doc) Quick start table

\item
start description of option in next line if option name too long

\item
fix \hypertarget{ltop}{top} in CSS (IE...)
\end{itemize}
\begin{itemize}
\item
REMEMBER ABOUT UPDATING DOCS
\end{itemize}

Copyright (c) Mikolaj Machowski, 2006

%%.. vim:set tw=78 ai fo+=n fo-=l:
\end{document}

